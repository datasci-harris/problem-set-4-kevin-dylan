% Options for packages loaded elsewhere
\PassOptionsToPackage{unicode}{hyperref}
\PassOptionsToPackage{hyphens}{url}
\PassOptionsToPackage{dvipsnames,svgnames,x11names}{xcolor}
%
\documentclass[
  letterpaper,
  DIV=11,
  numbers=noendperiod]{scrartcl}

\usepackage{amsmath,amssymb}
\usepackage{iftex}
\ifPDFTeX
  \usepackage[T1]{fontenc}
  \usepackage[utf8]{inputenc}
  \usepackage{textcomp} % provide euro and other symbols
\else % if luatex or xetex
  \usepackage{unicode-math}
  \defaultfontfeatures{Scale=MatchLowercase}
  \defaultfontfeatures[\rmfamily]{Ligatures=TeX,Scale=1}
\fi
\usepackage{lmodern}
\ifPDFTeX\else  
    % xetex/luatex font selection
\fi
% Use upquote if available, for straight quotes in verbatim environments
\IfFileExists{upquote.sty}{\usepackage{upquote}}{}
\IfFileExists{microtype.sty}{% use microtype if available
  \usepackage[]{microtype}
  \UseMicrotypeSet[protrusion]{basicmath} % disable protrusion for tt fonts
}{}
\makeatletter
\@ifundefined{KOMAClassName}{% if non-KOMA class
  \IfFileExists{parskip.sty}{%
    \usepackage{parskip}
  }{% else
    \setlength{\parindent}{0pt}
    \setlength{\parskip}{6pt plus 2pt minus 1pt}}
}{% if KOMA class
  \KOMAoptions{parskip=half}}
\makeatother
\usepackage{xcolor}
\setlength{\emergencystretch}{3em} % prevent overfull lines
\setcounter{secnumdepth}{-\maxdimen} % remove section numbering
% Make \paragraph and \subparagraph free-standing
\makeatletter
\ifx\paragraph\undefined\else
  \let\oldparagraph\paragraph
  \renewcommand{\paragraph}{
    \@ifstar
      \xxxParagraphStar
      \xxxParagraphNoStar
  }
  \newcommand{\xxxParagraphStar}[1]{\oldparagraph*{#1}\mbox{}}
  \newcommand{\xxxParagraphNoStar}[1]{\oldparagraph{#1}\mbox{}}
\fi
\ifx\subparagraph\undefined\else
  \let\oldsubparagraph\subparagraph
  \renewcommand{\subparagraph}{
    \@ifstar
      \xxxSubParagraphStar
      \xxxSubParagraphNoStar
  }
  \newcommand{\xxxSubParagraphStar}[1]{\oldsubparagraph*{#1}\mbox{}}
  \newcommand{\xxxSubParagraphNoStar}[1]{\oldsubparagraph{#1}\mbox{}}
\fi
\makeatother

\usepackage{color}
\usepackage{fancyvrb}
\newcommand{\VerbBar}{|}
\newcommand{\VERB}{\Verb[commandchars=\\\{\}]}
\DefineVerbatimEnvironment{Highlighting}{Verbatim}{commandchars=\\\{\}}
% Add ',fontsize=\small' for more characters per line
\usepackage{framed}
\definecolor{shadecolor}{RGB}{241,243,245}
\newenvironment{Shaded}{\begin{snugshade}}{\end{snugshade}}
\newcommand{\AlertTok}[1]{\textcolor[rgb]{0.68,0.00,0.00}{#1}}
\newcommand{\AnnotationTok}[1]{\textcolor[rgb]{0.37,0.37,0.37}{#1}}
\newcommand{\AttributeTok}[1]{\textcolor[rgb]{0.40,0.45,0.13}{#1}}
\newcommand{\BaseNTok}[1]{\textcolor[rgb]{0.68,0.00,0.00}{#1}}
\newcommand{\BuiltInTok}[1]{\textcolor[rgb]{0.00,0.23,0.31}{#1}}
\newcommand{\CharTok}[1]{\textcolor[rgb]{0.13,0.47,0.30}{#1}}
\newcommand{\CommentTok}[1]{\textcolor[rgb]{0.37,0.37,0.37}{#1}}
\newcommand{\CommentVarTok}[1]{\textcolor[rgb]{0.37,0.37,0.37}{\textit{#1}}}
\newcommand{\ConstantTok}[1]{\textcolor[rgb]{0.56,0.35,0.01}{#1}}
\newcommand{\ControlFlowTok}[1]{\textcolor[rgb]{0.00,0.23,0.31}{\textbf{#1}}}
\newcommand{\DataTypeTok}[1]{\textcolor[rgb]{0.68,0.00,0.00}{#1}}
\newcommand{\DecValTok}[1]{\textcolor[rgb]{0.68,0.00,0.00}{#1}}
\newcommand{\DocumentationTok}[1]{\textcolor[rgb]{0.37,0.37,0.37}{\textit{#1}}}
\newcommand{\ErrorTok}[1]{\textcolor[rgb]{0.68,0.00,0.00}{#1}}
\newcommand{\ExtensionTok}[1]{\textcolor[rgb]{0.00,0.23,0.31}{#1}}
\newcommand{\FloatTok}[1]{\textcolor[rgb]{0.68,0.00,0.00}{#1}}
\newcommand{\FunctionTok}[1]{\textcolor[rgb]{0.28,0.35,0.67}{#1}}
\newcommand{\ImportTok}[1]{\textcolor[rgb]{0.00,0.46,0.62}{#1}}
\newcommand{\InformationTok}[1]{\textcolor[rgb]{0.37,0.37,0.37}{#1}}
\newcommand{\KeywordTok}[1]{\textcolor[rgb]{0.00,0.23,0.31}{\textbf{#1}}}
\newcommand{\NormalTok}[1]{\textcolor[rgb]{0.00,0.23,0.31}{#1}}
\newcommand{\OperatorTok}[1]{\textcolor[rgb]{0.37,0.37,0.37}{#1}}
\newcommand{\OtherTok}[1]{\textcolor[rgb]{0.00,0.23,0.31}{#1}}
\newcommand{\PreprocessorTok}[1]{\textcolor[rgb]{0.68,0.00,0.00}{#1}}
\newcommand{\RegionMarkerTok}[1]{\textcolor[rgb]{0.00,0.23,0.31}{#1}}
\newcommand{\SpecialCharTok}[1]{\textcolor[rgb]{0.37,0.37,0.37}{#1}}
\newcommand{\SpecialStringTok}[1]{\textcolor[rgb]{0.13,0.47,0.30}{#1}}
\newcommand{\StringTok}[1]{\textcolor[rgb]{0.13,0.47,0.30}{#1}}
\newcommand{\VariableTok}[1]{\textcolor[rgb]{0.07,0.07,0.07}{#1}}
\newcommand{\VerbatimStringTok}[1]{\textcolor[rgb]{0.13,0.47,0.30}{#1}}
\newcommand{\WarningTok}[1]{\textcolor[rgb]{0.37,0.37,0.37}{\textit{#1}}}

\providecommand{\tightlist}{%
  \setlength{\itemsep}{0pt}\setlength{\parskip}{0pt}}\usepackage{longtable,booktabs,array}
\usepackage{calc} % for calculating minipage widths
% Correct order of tables after \paragraph or \subparagraph
\usepackage{etoolbox}
\makeatletter
\patchcmd\longtable{\par}{\if@noskipsec\mbox{}\fi\par}{}{}
\makeatother
% Allow footnotes in longtable head/foot
\IfFileExists{footnotehyper.sty}{\usepackage{footnotehyper}}{\usepackage{footnote}}
\makesavenoteenv{longtable}
\usepackage{graphicx}
\makeatletter
\def\maxwidth{\ifdim\Gin@nat@width>\linewidth\linewidth\else\Gin@nat@width\fi}
\def\maxheight{\ifdim\Gin@nat@height>\textheight\textheight\else\Gin@nat@height\fi}
\makeatother
% Scale images if necessary, so that they will not overflow the page
% margins by default, and it is still possible to overwrite the defaults
% using explicit options in \includegraphics[width, height, ...]{}
\setkeys{Gin}{width=\maxwidth,height=\maxheight,keepaspectratio}
% Set default figure placement to htbp
\makeatletter
\def\fps@figure{htbp}
\makeatother

\usepackage{fvextra}
\DefineVerbatimEnvironment{Highlighting}{Verbatim}{breaklines,commandchars=\\\{\}}
\KOMAoption{captions}{tableheading}
\makeatletter
\@ifpackageloaded{caption}{}{\usepackage{caption}}
\AtBeginDocument{%
\ifdefined\contentsname
  \renewcommand*\contentsname{Table of contents}
\else
  \newcommand\contentsname{Table of contents}
\fi
\ifdefined\listfigurename
  \renewcommand*\listfigurename{List of Figures}
\else
  \newcommand\listfigurename{List of Figures}
\fi
\ifdefined\listtablename
  \renewcommand*\listtablename{List of Tables}
\else
  \newcommand\listtablename{List of Tables}
\fi
\ifdefined\figurename
  \renewcommand*\figurename{Figure}
\else
  \newcommand\figurename{Figure}
\fi
\ifdefined\tablename
  \renewcommand*\tablename{Table}
\else
  \newcommand\tablename{Table}
\fi
}
\@ifpackageloaded{float}{}{\usepackage{float}}
\floatstyle{ruled}
\@ifundefined{c@chapter}{\newfloat{codelisting}{h}{lop}}{\newfloat{codelisting}{h}{lop}[chapter]}
\floatname{codelisting}{Listing}
\newcommand*\listoflistings{\listof{codelisting}{List of Listings}}
\makeatother
\makeatletter
\makeatother
\makeatletter
\@ifpackageloaded{caption}{}{\usepackage{caption}}
\@ifpackageloaded{subcaption}{}{\usepackage{subcaption}}
\makeatother

\ifLuaTeX
  \usepackage{selnolig}  % disable illegal ligatures
\fi
\usepackage{bookmark}

\IfFileExists{xurl.sty}{\usepackage{xurl}}{} % add URL line breaks if available
\urlstyle{same} % disable monospaced font for URLs
\hypersetup{
  pdftitle={PS4},
  colorlinks=true,
  linkcolor={blue},
  filecolor={Maroon},
  citecolor={Blue},
  urlcolor={Blue},
  pdfcreator={LaTeX via pandoc}}


\title{PS4}
\author{}
\date{}

\begin{document}
\maketitle

\RecustomVerbatimEnvironment{verbatim}{Verbatim}{
  showspaces = false,
  showtabs = false,
  breaksymbolleft={},
  breaklines
}


\subsection{Style Points (10 pts)}\label{style-points-10-pts}

Have refered to the minilesson on code style

\subsection{Submission Steps (10 pts)}\label{submission-steps-10-pts}

• Partner 1 (name and cnet ID): Qiyin Yao; qiyin • Partner 2 (name and
cnet ID):Dizhe Xia; dizhexia

``This submission is our work alone and complies with the 30538
integrity policy.'' Add your initials to indicate your agreement:
\textbf{\emph{QY}} \textbf{\emph{DX}}

Late coins used this pset: \textbf{\emph{1}} Late coins left after
submission: \textbf{\emph{2}}

\subsection{Download and explore the Provider of Services (POS) file (10
pts)}\label{download-and-explore-the-provider-of-services-pos-file-10-pts}

\begin{enumerate}
\def\labelenumi{\arabic{enumi}.}
\item
  I chose variables: 1. Provider Category Code: PRVDR\_CTGRY\_CD; 2.
  Provider Category Subtype Code: PRVDR\_CTGRY\_SBTYP\_CD; 3. Facility
  Name: FAC\_NAME; 4. Termination Code:PGM\_TRMNTN\_CD; 5. Address:
  City:CITY\_NAME; 6. State Abbreviation:STATE\_CD; 7. Certification
  Date:CRTFCTN\_DT.
\item
\end{enumerate}

\begin{enumerate}
\def\labelenumi{\alph{enumi}.}
\tightlist
\item
\end{enumerate}

\begin{Shaded}
\begin{Highlighting}[]
\ImportTok{import}\NormalTok{ pandas }\ImportTok{as}\NormalTok{ pd}

\CommentTok{\#import the data of pos2016}
\NormalTok{pos2016 }\OperatorTok{=}\NormalTok{ pd.read\_csv(}\StringTok{"/Users/xiadizhe/Documents/GitHub/problem{-}set{-}4{-}kevin{-}dylan/pos2016.csv"}\NormalTok{)}

\CommentTok{\#filter short term hospitals}
\NormalTok{short\_term\_hospitals }\OperatorTok{=}\NormalTok{ pos2016[(pos2016[}\StringTok{\textquotesingle{}PRVDR\_CTGRY\_CD\textquotesingle{}}\NormalTok{] }\OperatorTok{==} \FloatTok{1.0}\NormalTok{) }\OperatorTok{\&}\NormalTok{ (pos2016[}\StringTok{\textquotesingle{}PRVDR\_CTGRY\_SBTYP\_CD\textquotesingle{}}\NormalTok{] }\OperatorTok{==} \FloatTok{1.0}\NormalTok{)]}

\CommentTok{\#print number of short term hospitals}
\NormalTok{num\_short\_term\_hospitals }\OperatorTok{=}\NormalTok{ short\_term\_hospitals.shape[}\DecValTok{0}\NormalTok{]}
\BuiltInTok{print}\NormalTok{(}\SpecialStringTok{f"Number of short term hospital: }\SpecialCharTok{\{}\NormalTok{num\_short\_term\_hospitals}\SpecialCharTok{\}}\SpecialStringTok{"}\NormalTok{)}
\end{Highlighting}
\end{Shaded}

\begin{verbatim}
Number of short term hospital: 7245
\end{verbatim}

\begin{enumerate}
\def\labelenumi{\alph{enumi}.}
\setcounter{enumi}{1}
\tightlist
\item
  The number of short term hospital is 7,245. I do not think the number
  is sensible. As the article mentioned that ``There are nearly 5,000
  short-term, acute care hospitals in the United States''. Although the
  article was publised at mid 2016. But I do not think it is sensible to
  have such a huge increasing in short-term.
\end{enumerate}

\begin{enumerate}
\def\labelenumi{\arabic{enumi}.}
\setcounter{enumi}{2}
\tightlist
\item
\end{enumerate}

\begin{Shaded}
\begin{Highlighting}[]
\ImportTok{import}\NormalTok{ pandas }\ImportTok{as}\NormalTok{ pd}
\ImportTok{import}\NormalTok{ matplotlib.pyplot }\ImportTok{as}\NormalTok{ plt}

\CommentTok{\# Combine file paths for each year}
\NormalTok{file\_paths }\OperatorTok{=}\NormalTok{ \{}
    \DecValTok{2016}\NormalTok{: }\StringTok{"/Users/xiadizhe/Documents/GitHub/problem{-}set{-}4{-}kevin{-}dylan/pos2016.csv"}\NormalTok{,}
    \DecValTok{2017}\NormalTok{: }\StringTok{"/Users/xiadizhe/Documents/GitHub/problem{-}set{-}4{-}kevin{-}dylan/pos2017.csv"}\NormalTok{,}
    \DecValTok{2018}\NormalTok{: }\StringTok{"/Users/xiadizhe/Documents/GitHub/problem{-}set{-}4{-}kevin{-}dylan/pos2018.csv"}\NormalTok{,}
    \DecValTok{2019}\NormalTok{: }\StringTok{"/Users/xiadizhe/Documents/GitHub/problem{-}set{-}4{-}kevin{-}dylan/pos2019.csv"}
\NormalTok{\}}

\CommentTok{\# Dictionary to store the count of observations by year}
\NormalTok{observations\_by\_year }\OperatorTok{=}\NormalTok{ \{\}}

\CommentTok{\# Loop through each year and file path}
\ControlFlowTok{for}\NormalTok{ year, path }\KeywordTok{in}\NormalTok{ file\_paths.items():}
    
\NormalTok{    data }\OperatorTok{=}\NormalTok{ pd.read\_csv(path, encoding}\OperatorTok{=}\StringTok{\textquotesingle{}ISO{-}8859{-}1\textquotesingle{}}\NormalTok{)}
    
\NormalTok{    short\_term\_hospitals }\OperatorTok{=}\NormalTok{ data[(data[}\StringTok{\textquotesingle{}PRVDR\_CTGRY\_CD\textquotesingle{}}\NormalTok{] }\OperatorTok{==} \FloatTok{1.0}\NormalTok{) }\OperatorTok{\&}\NormalTok{ (data[}\StringTok{\textquotesingle{}PRVDR\_CTGRY\_SBTYP\_CD\textquotesingle{}}\NormalTok{] }\OperatorTok{==} \FloatTok{1.0}\NormalTok{)]}
    
\NormalTok{    observations\_by\_year[year] }\OperatorTok{=}\NormalTok{ short\_term\_hospitals.shape[}\DecValTok{0}\NormalTok{]}
    
    \CommentTok{\# Print the number of short{-}term hospitals for each year}
    \BuiltInTok{print}\NormalTok{(}\SpecialStringTok{f"Year }\SpecialCharTok{\{}\NormalTok{year}\SpecialCharTok{\}}\SpecialStringTok{: }\SpecialCharTok{\{}\NormalTok{observations\_by\_year[year]}\SpecialCharTok{\}}\SpecialStringTok{ short{-}term hospitals"}\NormalTok{)}

\CommentTok{\# Convert to a pandas Series for easy plotting}
\NormalTok{observations\_series }\OperatorTok{=}\NormalTok{ pd.Series(observations\_by\_year)}

\CommentTok{\# Plotting}
\NormalTok{plt.figure(figsize}\OperatorTok{=}\NormalTok{(}\DecValTok{8}\NormalTok{, }\DecValTok{6}\NormalTok{))}
\NormalTok{observations\_series.plot(kind}\OperatorTok{=}\StringTok{\textquotesingle{}bar\textquotesingle{}}\NormalTok{, color}\OperatorTok{=}\StringTok{\textquotesingle{}orange\textquotesingle{}}\NormalTok{)}
\NormalTok{plt.ylim(}\DecValTok{6950}\NormalTok{, }\DecValTok{8000}\NormalTok{)  }
\NormalTok{plt.xticks(rotation}\OperatorTok{=}\DecValTok{45}\NormalTok{)}
\NormalTok{plt.xlabel(}\StringTok{\textquotesingle{}Year\textquotesingle{}}\NormalTok{)}
\NormalTok{plt.ylabel(}\StringTok{\textquotesingle{}Number of Observations\textquotesingle{}}\NormalTok{)}
\NormalTok{plt.title(}\StringTok{\textquotesingle{}Number of Short{-}Term Hospital Observations by Year\textquotesingle{}}\NormalTok{)}
\NormalTok{plt.show()}
\end{Highlighting}
\end{Shaded}

\begin{verbatim}
Year 2016: 7245 short-term hospitals
Year 2017: 7260 short-term hospitals
Year 2018: 7277 short-term hospitals
Year 2019: 7303 short-term hospitals
\end{verbatim}

\includegraphics{pset4_template_files/figure-pdf/cell-3-output-2.pdf}

\begin{enumerate}
\def\labelenumi{\arabic{enumi}.}
\setcounter{enumi}{3}
\tightlist
\item
\end{enumerate}

\begin{enumerate}
\def\labelenumi{\alph{enumi}.}
\tightlist
\item
\end{enumerate}

\begin{Shaded}
\begin{Highlighting}[]
\ImportTok{import}\NormalTok{ pandas }\ImportTok{as}\NormalTok{ pd}
\ImportTok{import}\NormalTok{ matplotlib.pyplot }\ImportTok{as}\NormalTok{ plt}

\CommentTok{\# Define file paths for each year}
\NormalTok{file\_paths }\OperatorTok{=}\NormalTok{ \{}
    \DecValTok{2016}\NormalTok{: }\StringTok{"/Users/xiadizhe/Documents/GitHub/problem{-}set{-}4{-}kevin{-}dylan/pos2016.csv"}\NormalTok{,}
    \DecValTok{2017}\NormalTok{: }\StringTok{"/Users/xiadizhe/Documents/GitHub/problem{-}set{-}4{-}kevin{-}dylan/pos2017.csv"}\NormalTok{,}
    \DecValTok{2018}\NormalTok{: }\StringTok{"/Users/xiadizhe/Documents/GitHub/problem{-}set{-}4{-}kevin{-}dylan/pos2018.csv"}\NormalTok{,}
    \DecValTok{2019}\NormalTok{: }\StringTok{"/Users/xiadizhe/Documents/GitHub/problem{-}set{-}4{-}kevin{-}dylan/pos2019.csv"}
\NormalTok{\}}

\CommentTok{\# Dictionary to store unique hospital counts by year}
\NormalTok{unique\_hospitals\_by\_year }\OperatorTok{=}\NormalTok{ \{\}}

\CommentTok{\# Loop through each file to calculate the unique hospital counts}
\ControlFlowTok{for}\NormalTok{ year, path }\KeywordTok{in}\NormalTok{ file\_paths.items():}
\NormalTok{    data }\OperatorTok{=}\NormalTok{ pd.read\_csv(path, encoding}\OperatorTok{=}\StringTok{\textquotesingle{}ISO{-}8859{-}1\textquotesingle{}}\NormalTok{)}
    
\NormalTok{    short\_term\_hospitals }\OperatorTok{=}\NormalTok{ data[(data[}\StringTok{\textquotesingle{}PRVDR\_CTGRY\_CD\textquotesingle{}}\NormalTok{] }\OperatorTok{==} \FloatTok{1.0}\NormalTok{) }\OperatorTok{\&}\NormalTok{ (data[}\StringTok{\textquotesingle{}PRVDR\_CTGRY\_SBTYP\_CD\textquotesingle{}}\NormalTok{] }\OperatorTok{==} \FloatTok{1.0}\NormalTok{)]}
    
\NormalTok{    unique\_count }\OperatorTok{=}\NormalTok{ short\_term\_hospitals[}\StringTok{\textquotesingle{}PRVDR\_NUM\textquotesingle{}}\NormalTok{].nunique()}
\NormalTok{    unique\_hospitals\_by\_year[year] }\OperatorTok{=}\NormalTok{ unique\_count}
    
    \CommentTok{\# Print unique hospital count per year}
    \BuiltInTok{print}\NormalTok{(}\SpecialStringTok{f"Year }\SpecialCharTok{\{}\NormalTok{year}\SpecialCharTok{\}}\SpecialStringTok{: }\SpecialCharTok{\{}\NormalTok{unique\_count}\SpecialCharTok{\}}\SpecialStringTok{ unique short{-}term hospitals"}\NormalTok{)}

\CommentTok{\# Convert to Series for plotting}
\NormalTok{unique\_hospitals\_series }\OperatorTok{=}\NormalTok{ pd.Series(unique\_hospitals\_by\_year)}

\CommentTok{\# Plotting}
\NormalTok{plt.figure(figsize}\OperatorTok{=}\NormalTok{(}\DecValTok{8}\NormalTok{, }\DecValTok{6}\NormalTok{))}
\NormalTok{unique\_hospitals\_series.plot(kind}\OperatorTok{=}\StringTok{\textquotesingle{}bar\textquotesingle{}}\NormalTok{, color}\OperatorTok{=}\StringTok{\textquotesingle{}skyblue\textquotesingle{}}\NormalTok{)}
\NormalTok{plt.ylim(}\DecValTok{6950}\NormalTok{, }\DecValTok{8000}\NormalTok{)  }
\NormalTok{plt.xticks(rotation}\OperatorTok{=}\DecValTok{45}\NormalTok{)}
\NormalTok{plt.xlabel(}\StringTok{\textquotesingle{}Year\textquotesingle{}}\NormalTok{)}
\NormalTok{plt.ylabel(}\StringTok{\textquotesingle{}Number of Unique Hospitals\textquotesingle{}}\NormalTok{)}
\NormalTok{plt.title(}\StringTok{\textquotesingle{}Number of Unique Short{-}Term Hospitals by Year\textquotesingle{}}\NormalTok{)}
\NormalTok{plt.show()}
\end{Highlighting}
\end{Shaded}

\begin{verbatim}
Year 2016: 7245 unique short-term hospitals
Year 2017: 7260 unique short-term hospitals
Year 2018: 7277 unique short-term hospitals
Year 2019: 7303 unique short-term hospitals
\end{verbatim}

\includegraphics{pset4_template_files/figure-pdf/cell-4-output-2.pdf}

\begin{enumerate}
\def\labelenumi{\alph{enumi}.}
\setcounter{enumi}{1}
\tightlist
\item
  By comparison to the outcomes and plot of previous step. I can find
  out that the plots are fully the same. As the number of unique
  hospitals (based on short-term hospital) and the number of the
  short-term hospital are the same across four years.
\end{enumerate}

\subsection{Identify hospital closures in POS file (15 pts)
(*)}\label{identify-hospital-closures-in-pos-file-15-pts}

\subsection{1.}\label{section}

\begin{Shaded}
\begin{Highlighting}[]
\ImportTok{import}\NormalTok{ pandas }\ImportTok{as}\NormalTok{ pd}

\CommentTok{\# Suppress SettingWithCopyWarning}
\NormalTok{pd.options.mode.chained\_assignment }\OperatorTok{=} \VariableTok{None}
\CommentTok{\# Your existing code}
\ImportTok{import}\NormalTok{ time}
\CommentTok{\# Define file paths}
\NormalTok{files }\OperatorTok{=}\NormalTok{ \{}
    \DecValTok{2016}\NormalTok{: }\StringTok{\textquotesingle{}/Users/xiadizhe/Documents/GitHub/problem{-}set{-}4{-}kevin{-}dylan/pos2016.csv\textquotesingle{}}\NormalTok{,}
    \DecValTok{2017}\NormalTok{: }\StringTok{\textquotesingle{}/Users/xiadizhe/Documents/GitHub/problem{-}set{-}4{-}kevin{-}dylan/pos2017.csv\textquotesingle{}}\NormalTok{,}
    \DecValTok{2018}\NormalTok{: }\StringTok{\textquotesingle{}/Users/xiadizhe/Documents/GitHub/problem{-}set{-}4{-}kevin{-}dylan/pos2018.csv\textquotesingle{}}\NormalTok{,}
    \DecValTok{2019}\NormalTok{: }\StringTok{\textquotesingle{}/Users/xiadizhe/Documents/GitHub/problem{-}set{-}4{-}kevin{-}dylan/pos2019.csv\textquotesingle{}}
\NormalTok{\}}

\CommentTok{\# Store data for each year}
\NormalTok{data\_by\_year }\OperatorTok{=}\NormalTok{ \{\}}

\CommentTok{\# Load data for each year and filter short{-}term hospitals (PRVDR\_CTGRY\_CD and PRVDR\_CTGRY\_SBTYP\_CD both equal to 1)}
\ControlFlowTok{for}\NormalTok{ year, }\BuiltInTok{file} \KeywordTok{in}\NormalTok{ files.items():}
\NormalTok{    data }\OperatorTok{=}\NormalTok{ pd.read\_csv(}\BuiltInTok{file}\NormalTok{, encoding}\OperatorTok{=}\StringTok{\textquotesingle{}ISO{-}8859{-}1\textquotesingle{}}\NormalTok{)}
\NormalTok{    data\_filtered }\OperatorTok{=}\NormalTok{ data[(data[}\StringTok{\textquotesingle{}PRVDR\_CTGRY\_CD\textquotesingle{}}\NormalTok{] }\OperatorTok{==} \DecValTok{1}\NormalTok{) }\OperatorTok{\&}\NormalTok{ (data[}\StringTok{\textquotesingle{}PRVDR\_CTGRY\_SBTYP\_CD\textquotesingle{}}\NormalTok{] }\OperatorTok{==} \DecValTok{1}\NormalTok{)]}
\NormalTok{    data\_filtered[}\StringTok{\textquotesingle{}Year\textquotesingle{}}\NormalTok{] }\OperatorTok{=}\NormalTok{ year  }\CommentTok{\# Add year column for easier identification later}
\NormalTok{    data\_by\_year[year] }\OperatorTok{=}\NormalTok{ data\_filtered[[}\StringTok{\textquotesingle{}PRVDR\_NUM\textquotesingle{}}\NormalTok{, }\StringTok{\textquotesingle{}FAC\_NAME\textquotesingle{}}\NormalTok{, }\StringTok{\textquotesingle{}CITY\_NAME\textquotesingle{}}\NormalTok{, }\StringTok{\textquotesingle{}STATE\_CD\textquotesingle{}}\NormalTok{, }\StringTok{\textquotesingle{}PGM\_TRMNTN\_CD\textquotesingle{}}\NormalTok{, }\StringTok{\textquotesingle{}ZIP\_CD\textquotesingle{}}\NormalTok{, }\StringTok{\textquotesingle{}Year\textquotesingle{}}\NormalTok{]]}

\CommentTok{\# Combine all years into a single DataFrame}
\NormalTok{df\_combine }\OperatorTok{=}\NormalTok{ pd.concat(data\_by\_year.values())}

\CommentTok{\# Get active hospitals in 2016}
\NormalTok{active\_2016 }\OperatorTok{=}\NormalTok{ df\_combine[(df\_combine[}\StringTok{\textquotesingle{}Year\textquotesingle{}}\NormalTok{] }\OperatorTok{==} \DecValTok{2016}\NormalTok{) }\OperatorTok{\&}\NormalTok{ (df\_combine[}\StringTok{\textquotesingle{}PGM\_TRMNTN\_CD\textquotesingle{}}\NormalTok{] }\OperatorTok{==} \DecValTok{0}\NormalTok{)]}

\CommentTok{\# Initialize a list to store suspected closures}
\NormalTok{suspected\_closures\_data }\OperatorTok{=}\NormalTok{ []}

\CommentTok{\# Check for closures by iterating from 2017 to 2019}
\ControlFlowTok{for}\NormalTok{ year }\KeywordTok{in} \BuiltInTok{range}\NormalTok{(}\DecValTok{2017}\NormalTok{, }\DecValTok{2020}\NormalTok{):  }\CommentTok{\# Starting from 2017}
\NormalTok{    current\_year\_data }\OperatorTok{=}\NormalTok{ df\_combine[df\_combine[}\StringTok{\textquotesingle{}Year\textquotesingle{}}\NormalTok{] }\OperatorTok{==}\NormalTok{ year]}
\NormalTok{    active\_this\_year }\OperatorTok{=} \BuiltInTok{set}\NormalTok{(current\_year\_data[current\_year\_data[}\StringTok{\textquotesingle{}PGM\_TRMNTN\_CD\textquotesingle{}}\NormalTok{] }\OperatorTok{==} \DecValTok{0}\NormalTok{][}\StringTok{\textquotesingle{}PRVDR\_NUM\textquotesingle{}}\NormalTok{])}

    \ControlFlowTok{for}\NormalTok{ index, hospital }\KeywordTok{in}\NormalTok{ active\_2016.iterrows():}
        \ControlFlowTok{if}\NormalTok{ hospital[}\StringTok{\textquotesingle{}PRVDR\_NUM\textquotesingle{}}\NormalTok{] }\KeywordTok{not} \KeywordTok{in}\NormalTok{ active\_this\_year:}
            \CommentTok{\# Get the termination code for the hospital if it exists in the closure year dataset}
\NormalTok{            termination\_code }\OperatorTok{=} \StringTok{\textquotesingle{}Unknown\textquotesingle{}}
\NormalTok{            matching\_hospitals }\OperatorTok{=}\NormalTok{ current\_year\_data[current\_year\_data[}\StringTok{\textquotesingle{}PRVDR\_NUM\textquotesingle{}}\NormalTok{] }\OperatorTok{==}\NormalTok{ hospital[}\StringTok{\textquotesingle{}PRVDR\_NUM\textquotesingle{}}\NormalTok{]]}
            \ControlFlowTok{if} \KeywordTok{not}\NormalTok{ matching\_hospitals.empty:}
\NormalTok{                termination\_code }\OperatorTok{=}\NormalTok{ matching\_hospitals[}\StringTok{\textquotesingle{}PGM\_TRMNTN\_CD\textquotesingle{}}\NormalTok{].iloc[}\DecValTok{0}\NormalTok{]}

            \CommentTok{\# Store data as a dictionary and append to the list}
\NormalTok{            suspected\_closures\_data.append(\{}
                \StringTok{\textquotesingle{}PRVDR\_NUM\textquotesingle{}}\NormalTok{: hospital[}\StringTok{\textquotesingle{}PRVDR\_NUM\textquotesingle{}}\NormalTok{],}
                \StringTok{\textquotesingle{}FAC\_NAME\textquotesingle{}}\NormalTok{: hospital[}\StringTok{\textquotesingle{}FAC\_NAME\textquotesingle{}}\NormalTok{],}
                \StringTok{\textquotesingle{}ZIP\_CD\textquotesingle{}}\NormalTok{: hospital[}\StringTok{\textquotesingle{}ZIP\_CD\textquotesingle{}}\NormalTok{],}
                \StringTok{\textquotesingle{}Year of Closure\textquotesingle{}}\NormalTok{: year,}
                \StringTok{\textquotesingle{}Termination Code\textquotesingle{}}\NormalTok{: termination\_code}
\NormalTok{            \})}

\CommentTok{\# Create a DataFrame from the list of dictionaries}
\NormalTok{suspected\_closures }\OperatorTok{=}\NormalTok{ pd.DataFrame(suspected\_closures\_data)}

\CommentTok{\# Remove duplicate entries in case a hospital is recorded more than once}
\NormalTok{predicted\_closures }\OperatorTok{=}\NormalTok{ suspected\_closures.drop\_duplicates(subset}\OperatorTok{=}\StringTok{\textquotesingle{}PRVDR\_NUM\textquotesingle{}}\NormalTok{)}

\CommentTok{\# Display the results}
\BuiltInTok{print}\NormalTok{(}\SpecialStringTok{f"Total suspected closures: }\SpecialCharTok{\{}\BuiltInTok{len}\NormalTok{(predicted\_closures)}\SpecialCharTok{\}}\SpecialStringTok{"}\NormalTok{)}
\end{Highlighting}
\end{Shaded}

\begin{verbatim}
Total suspected closures: 174
\end{verbatim}

\subsection{2.}\label{section-1}

\begin{Shaded}
\begin{Highlighting}[]
\CommentTok{\# Sort the predicted closures by hospital name (FAC\_NAME) and select the first 10 rows}
\NormalTok{sorted\_closures }\OperatorTok{=}\NormalTok{ predicted\_closures.sort\_values(by}\OperatorTok{=}\StringTok{\textquotesingle{}FAC\_NAME\textquotesingle{}}\NormalTok{).head(}\DecValTok{10}\NormalTok{)}

\CommentTok{\# Display the names and year of suspected closure for the first 10 rows}
\BuiltInTok{print}\NormalTok{(sorted\_closures[[}\StringTok{\textquotesingle{}FAC\_NAME\textquotesingle{}}\NormalTok{, }\StringTok{\textquotesingle{}Year of Closure\textquotesingle{}}\NormalTok{]])}
\end{Highlighting}
\end{Shaded}

\begin{verbatim}
                                          FAC_NAME  Year of Closure
0                           ABRAZO MARYVALE CAMPUS             2017
1        ADVENTIST MEDICAL CENTER - CENTRAL VALLEY             2017
90                         AFFINITY MEDICAL CENTER             2018
15   ALBANY MEDICAL CENTER / SOUTH CLINICAL CAMPUS             2017
29        ALLEGIANCE SPECIALTY HOSPITAL OF KILGORE             2017
200                        ALLIANCE LAIRD HOSPITAL             2019
239                       ALLIANCEHEALTH DEACONESS             2019
164                  ANNE BATES LEACH EYE HOSPITAL             2019
3          ARKANSAS VALLEY REGIONAL MEDICAL CENTER             2017
11             BANNER CHURCHILL COMMUNITY HOSPITAL             2017
\end{verbatim}

\subsection{3.}\label{section-2}

\subsubsection{a.}\label{a.}

\begin{Shaded}
\begin{Highlighting}[]
\CommentTok{\# Initialize a list to store suspected CMS certification numbers for potential mergers}
\NormalTok{potential\_mergers }\OperatorTok{=}\NormalTok{ []}

\CommentTok{\# Iterate through each row in the predicted\_closures DataFrame}
\ControlFlowTok{for}\NormalTok{ index, row }\KeywordTok{in}\NormalTok{ predicted\_closures.iterrows():}
    \CommentTok{\# Get the ZIP code and the suspected closure year for the current hospital}
\NormalTok{    zip\_code }\OperatorTok{=}\NormalTok{ row[}\StringTok{\textquotesingle{}ZIP\_CD\textquotesingle{}}\NormalTok{]}
\NormalTok{    closure\_year }\OperatorTok{=}\NormalTok{ row[}\StringTok{\textquotesingle{}Year of Closure\textquotesingle{}}\NormalTok{]}
    
    \CommentTok{\# Get data for the year after the suspected closure in the same ZIP code}
\NormalTok{    next\_year\_data }\OperatorTok{=}\NormalTok{ df\_combine[}
\NormalTok{        (df\_combine[}\StringTok{\textquotesingle{}Year\textquotesingle{}}\NormalTok{] }\OperatorTok{==}\NormalTok{ closure\_year }\OperatorTok{+} \DecValTok{1}\NormalTok{) }\OperatorTok{\&} 
\NormalTok{        (df\_combine[}\StringTok{\textquotesingle{}ZIP\_CD\textquotesingle{}}\NormalTok{] }\OperatorTok{==}\NormalTok{ zip\_code) }\OperatorTok{\&} 
\NormalTok{        (df\_combine[}\StringTok{\textquotesingle{}PGM\_TRMNTN\_CD\textquotesingle{}}\NormalTok{] }\OperatorTok{==} \DecValTok{0}\NormalTok{)}
\NormalTok{    ]}
    
    \CommentTok{\# Get data for the year of the suspected closure in the same ZIP code}
\NormalTok{    current\_year\_data }\OperatorTok{=}\NormalTok{ df\_combine[}
\NormalTok{        (df\_combine[}\StringTok{\textquotesingle{}Year\textquotesingle{}}\NormalTok{] }\OperatorTok{==}\NormalTok{ closure\_year) }\OperatorTok{\&} 
\NormalTok{        (df\_combine[}\StringTok{\textquotesingle{}ZIP\_CD\textquotesingle{}}\NormalTok{] }\OperatorTok{==}\NormalTok{ zip\_code) }\OperatorTok{\&} 
\NormalTok{        (df\_combine[}\StringTok{\textquotesingle{}PGM\_TRMNTN\_CD\textquotesingle{}}\NormalTok{] }\OperatorTok{==} \DecValTok{0}\NormalTok{)}
\NormalTok{    ]}
    
    \CommentTok{\# If the number of active hospitals in the next year is not less than the current year,}
    \CommentTok{\# it suggests a potential merger or acquisition, so add the hospital to potential\_mergers}
    \ControlFlowTok{if} \BuiltInTok{len}\NormalTok{(next\_year\_data) }\OperatorTok{\textgreater{}=} \BuiltInTok{len}\NormalTok{(current\_year\_data):}
\NormalTok{        potential\_mergers.append(row[}\StringTok{\textquotesingle{}PRVDR\_NUM\textquotesingle{}}\NormalTok{])}


\CommentTok{\# Display the count of suspected mergers/acquisitions}
\BuiltInTok{print}\NormalTok{(}\StringTok{"Number of hospitals potentially involved in a merger/acquisition:"}\NormalTok{, }\BuiltInTok{len}\NormalTok{(potential\_mergers))}
\end{Highlighting}
\end{Shaded}

\begin{verbatim}
Number of hospitals potentially involved in a merger/acquisition: 148
\end{verbatim}

\subsubsection{b.}\label{b.}

\begin{Shaded}
\begin{Highlighting}[]
\CommentTok{\# Filter out hospitals identified as potential mergers/acquisitions from predicted\_closures}
\NormalTok{hospital\_closures\_filtered }\OperatorTok{=}\NormalTok{ predicted\_closures[}\OperatorTok{\textasciitilde{}}\NormalTok{predicted\_closures[}\StringTok{\textquotesingle{}PRVDR\_NUM\textquotesingle{}}\NormalTok{].isin(potential\_mergers)]}

\BuiltInTok{print}\NormalTok{(}\SpecialStringTok{f"Number of hospital closures after excluding mergers/acquisitions: }\SpecialCharTok{\{}\BuiltInTok{len}\NormalTok{(hospital\_closures\_filtered)}\SpecialCharTok{\}}\SpecialStringTok{"}\NormalTok{)}
\end{Highlighting}
\end{Shaded}

\begin{verbatim}
Number of hospital closures after excluding mergers/acquisitions: 26
\end{verbatim}

\subsubsection{c.}\label{c.}

\begin{Shaded}
\begin{Highlighting}[]
\CommentTok{\# Sort the corrected closures by hospital name (FAC\_NAME) and select the first 10 rows}
\NormalTok{sorted\_corrected\_closures }\OperatorTok{=}\NormalTok{ hospital\_closures\_filtered.sort\_values(by}\OperatorTok{=}\StringTok{\textquotesingle{}FAC\_NAME\textquotesingle{}}\NormalTok{).head(}\DecValTok{10}\NormalTok{)}

\CommentTok{\# Display the names and year of suspected closure for the first 10 rows}
\BuiltInTok{print}\NormalTok{(sorted\_corrected\_closures[[}\StringTok{\textquotesingle{}FAC\_NAME\textquotesingle{}}\NormalTok{, }\StringTok{\textquotesingle{}Year of Closure\textquotesingle{}}\NormalTok{]])}
\end{Highlighting}
\end{Shaded}

\begin{verbatim}
                                              FAC_NAME  Year of Closure
239                           ALLIANCEHEALTH DEACONESS             2019
164                      ANNE BATES LEACH EYE HOSPITAL             2019
253                      BARIX CLINICS OF PENNSYLVANIA             2019
304  BAYLOR SCOTT & WHITE EMERGENCY MEDICAL CENTER ...             2019
203               BLACK RIVER COMMUNITY MEDICAL CENTER             2019
171                        CHESTATEE REGIONAL HOSPITAL             2019
191               DOCTORS HOSPITAL AT DEER CREEK L L C             2019
287                         EL PASO SPECIALTY HOSPITAL             2019
176                           FRANCISCAN HEALTH CARMEL             2019
168                           HUTCHESON MEDICAL CENTER             2019
\end{verbatim}

\subsection{Download Census zip code shapefile (10
pt)}\label{download-census-zip-code-shapefile-10-pt}

\begin{enumerate}
\def\labelenumi{\arabic{enumi}.}
\tightlist
\item
\end{enumerate}

\begin{enumerate}
\def\labelenumi{\alph{enumi}.}
\item
  The five files are DBF file, PRJ file, SHP file, SHX file and xmlfile.
  The DBF file is a database file attribute data associated with each
  ZCTA, including fields for unique ZCTA ID, name, area, and other
  relevant information.The SHX file is the index file, which facilitates
  indexing between the .dbf and .shp files, enabling efficient access to
  spatial and attribute data. The PRj file is the projection file,
  containing coordinate system and projection details. This file
  specifies that the data uses the North American Datum of 1983 (NAD83)
  in decimal degrees. The xml file is the metadata file describing the
  contents, purpose, usage restrictions, and accuracy of the dataset.
  The SHP file is a key component of the shapefile format, which
  Contains the actual shapes (geometry) of features.
\item
  After unzipping, the DBF file is 6,275 KB. The RPJ file is 1 KB. The
  SHP file is 817,915 KB. The SHX file is 259 KB. The xml file is 16 KB.
\end{enumerate}

\begin{enumerate}
\def\labelenumi{\arabic{enumi}.}
\setcounter{enumi}{1}
\tightlist
\item
\end{enumerate}

\begin{Shaded}
\begin{Highlighting}[]
\ImportTok{import}\NormalTok{ geopandas }\ImportTok{as}\NormalTok{ gpd}
\ImportTok{import}\NormalTok{ pandas }\ImportTok{as}\NormalTok{ pd}
\ImportTok{import}\NormalTok{ matplotlib.pyplot }\ImportTok{as}\NormalTok{ plt}

\CommentTok{\# Define file paths}
\NormalTok{shapefile\_path }\OperatorTok{=} \StringTok{"/Users/xiadizhe/Documents/GitHub/problem{-}set{-}4{-}kevin{-}dylan/gz\_2010\_us\_860\_00\_500k/gz\_2010\_us\_860\_00\_500k.shp"}
\NormalTok{pos2016\_path }\OperatorTok{=} \StringTok{"/Users/xiadizhe/Documents/GitHub/problem{-}set{-}4{-}kevin{-}dylan/pos2016.csv"}

\CommentTok{\# Load the shapefile}
\NormalTok{gdf }\OperatorTok{=}\NormalTok{ gpd.read\_file(shapefile\_path)}

\CommentTok{\# Filter to only Texas zip codes (ZCTA5 that starts with 75, 76, 77, 78, or 79)}
\NormalTok{gdf[}\StringTok{\textquotesingle{}ZCTA5\textquotesingle{}}\NormalTok{] }\OperatorTok{=}\NormalTok{ gdf[}\StringTok{\textquotesingle{}ZCTA5\textquotesingle{}}\NormalTok{].astype(}\BuiltInTok{str}\NormalTok{)}
\NormalTok{tx\_gdf }\OperatorTok{=}\NormalTok{ gdf[gdf[}\StringTok{\textquotesingle{}ZCTA5\textquotesingle{}}\NormalTok{].}\BuiltInTok{str}\NormalTok{.startswith((}\StringTok{\textquotesingle{}75\textquotesingle{}}\NormalTok{, }\StringTok{\textquotesingle{}76\textquotesingle{}}\NormalTok{, }\StringTok{\textquotesingle{}77\textquotesingle{}}\NormalTok{, }\StringTok{\textquotesingle{}78\textquotesingle{}}\NormalTok{, }\StringTok{\textquotesingle{}79\textquotesingle{}}\NormalTok{))]}

\CommentTok{\# Load the POS 2016 hospital data}
\NormalTok{pos\_2016\_df }\OperatorTok{=}\NormalTok{ pd.read\_csv(pos2016\_path, encoding}\OperatorTok{=}\StringTok{\textquotesingle{}ISO{-}8859{-}1\textquotesingle{}}\NormalTok{)}

\CommentTok{\# Filter short{-}term hospitals (category and subtype codes are both 1)}
\NormalTok{short\_term\_hospitals\_2016 }\OperatorTok{=}\NormalTok{ pos\_2016\_df[(pos\_2016\_df[}\StringTok{\textquotesingle{}PRVDR\_CTGRY\_CD\textquotesingle{}}\NormalTok{] }\OperatorTok{==} \FloatTok{1.0}\NormalTok{) }\OperatorTok{\&}\NormalTok{ (pos\_2016\_df[}\StringTok{\textquotesingle{}PRVDR\_CTGRY\_SBTYP\_CD\textquotesingle{}}\NormalTok{] }\OperatorTok{==} \FloatTok{1.0}\NormalTok{)]}

\CommentTok{\# Group by zip code and count the number of hospitals per zip code}
\NormalTok{hospital\_counts }\OperatorTok{=}\NormalTok{ short\_term\_hospitals\_2016.groupby(}\StringTok{\textquotesingle{}ZIP\_CD\textquotesingle{}}\NormalTok{).size().reset\_index(name}\OperatorTok{=}\StringTok{\textquotesingle{}hospital\_count\textquotesingle{}}\NormalTok{)}

\CommentTok{\# Convert ZIP\_CD to string and add leading zeros to match shapefile format}
\NormalTok{hospital\_counts[}\StringTok{\textquotesingle{}ZIP\_CD\textquotesingle{}}\NormalTok{] }\OperatorTok{=}\NormalTok{ hospital\_counts[}\StringTok{\textquotesingle{}ZIP\_CD\textquotesingle{}}\NormalTok{].}\BuiltInTok{apply}\NormalTok{(}\KeywordTok{lambda}\NormalTok{ x: }\BuiltInTok{str}\NormalTok{(}\BuiltInTok{int}\NormalTok{(x)).zfill(}\DecValTok{5}\NormalTok{))}

\CommentTok{\# Merge hospital counts with the Texas GeoDataFrame}
\NormalTok{tx\_gdf }\OperatorTok{=}\NormalTok{ tx\_gdf.merge(hospital\_counts, left\_on}\OperatorTok{=}\StringTok{\textquotesingle{}ZCTA5\textquotesingle{}}\NormalTok{, right\_on}\OperatorTok{=}\StringTok{\textquotesingle{}ZIP\_CD\textquotesingle{}}\NormalTok{, how}\OperatorTok{=}\StringTok{\textquotesingle{}left\textquotesingle{}}\NormalTok{)}

\CommentTok{\# Fill N/A values in \textquotesingle{}hospital\_count\textquotesingle{} with 0}
\NormalTok{tx\_gdf[}\StringTok{\textquotesingle{}hospital\_count\textquotesingle{}}\NormalTok{] }\OperatorTok{=}\NormalTok{ tx\_gdf[}\StringTok{\textquotesingle{}hospital\_count\textquotesingle{}}\NormalTok{].fillna(}\DecValTok{0}\NormalTok{)}

\CommentTok{\# Plot a choropleth map of the number of hospitals per zip code}
\NormalTok{fig, ax }\OperatorTok{=}\NormalTok{ plt.subplots(}\DecValTok{1}\NormalTok{, }\DecValTok{1}\NormalTok{, figsize}\OperatorTok{=}\NormalTok{(}\DecValTok{12}\NormalTok{, }\DecValTok{12}\NormalTok{))}
\NormalTok{tx\_gdf.plot(column}\OperatorTok{=}\StringTok{\textquotesingle{}hospital\_count\textquotesingle{}}\NormalTok{, cmap}\OperatorTok{=}\StringTok{\textquotesingle{}OrRd\textquotesingle{}}\NormalTok{, linewidth}\OperatorTok{=}\FloatTok{0.8}\NormalTok{, ax}\OperatorTok{=}\NormalTok{ax, edgecolor}\OperatorTok{=}\StringTok{\textquotesingle{}0.8\textquotesingle{}}\NormalTok{, legend}\OperatorTok{=}\VariableTok{True}\NormalTok{)}
\NormalTok{ax.set\_title(}\StringTok{\textquotesingle{}Number of Hospitals by Zip Code in Texas (2016)\textquotesingle{}}\NormalTok{)}
\NormalTok{ax.set\_axis\_off()}

\CommentTok{\# Show the plot}
\NormalTok{plt.show()}
\end{Highlighting}
\end{Shaded}

\includegraphics{pset4_template_files/figure-pdf/cell-10-output-1.pdf}

\subsection{Calculate zip code's distance to the nearest hospital (20
pts)
(*)}\label{calculate-zip-codes-distance-to-the-nearest-hospital-20-pts}

\subsection{1.}\label{section-3}

\begin{Shaded}
\begin{Highlighting}[]
\ImportTok{import}\NormalTok{ geopandas }\ImportTok{as}\NormalTok{ gpd}

\CommentTok{\# Define the file path for the ZIP code shapefile}
\NormalTok{shapefile\_path }\OperatorTok{=} \StringTok{"/Users/xiadizhe/Documents/GitHub/problem{-}set{-}4{-}kevin{-}dylan/gz\_2010\_us\_860\_00\_500k/gz\_2010\_us\_860\_00\_500k.shp"}

\CommentTok{\# Load the shapefile into a GeoDataFrame}
\NormalTok{zips\_all\_centroids }\OperatorTok{=}\NormalTok{ gpd.read\_file(shapefile\_path)}

\CommentTok{\# Calculate the centroid of each ZIP code area}
\NormalTok{zips\_all\_centroids[}\StringTok{\textquotesingle{}geometry\textquotesingle{}}\NormalTok{] }\OperatorTok{=}\NormalTok{ zips\_all\_centroids[}\StringTok{\textquotesingle{}geometry\textquotesingle{}}\NormalTok{].centroid}

\CommentTok{\# Display the dimensions and column names of the resulting GeoDataFrame}
\BuiltInTok{print}\NormalTok{(}\StringTok{"Dimensions of GeoDataFrame:"}\NormalTok{, zips\_all\_centroids.shape)}
\BuiltInTok{print}\NormalTok{(}\StringTok{"Columns in GeoDataFrame:"}\NormalTok{, zips\_all\_centroids.columns)}

\CommentTok{\# Show the first few rows to understand the data structure}
\BuiltInTok{print}\NormalTok{(zips\_all\_centroids.head())}
\end{Highlighting}
\end{Shaded}

\begin{verbatim}
/var/folders/_4/32019gfn3jscfp2y27sc0v180000gn/T/ipykernel_8908/3982728313.py:10: UserWarning:

Geometry is in a geographic CRS. Results from 'centroid' are likely incorrect. Use 'GeoSeries.to_crs()' to re-project geometries to a projected CRS before this operation.

\end{verbatim}

\begin{verbatim}
Dimensions of GeoDataFrame: (33120, 6)
Columns in GeoDataFrame: Index(['GEO_ID', 'ZCTA5', 'NAME', 'LSAD', 'CENSUSAREA', 'geometry'], dtype='object')
           GEO_ID  ZCTA5   NAME   LSAD  CENSUSAREA                    geometry
0  8600000US01040  01040  01040  ZCTA5      21.281  POINT (-72.64107 42.21257)
1  8600000US01050  01050  01050  ZCTA5      38.329  POINT (-72.86985 42.28786)
2  8600000US01053  01053  01053  ZCTA5       5.131  POINT (-72.71162 42.35349)
3  8600000US01056  01056  01056  ZCTA5      27.205  POINT (-72.45805 42.19215)
4  8600000US01057  01057  01057  ZCTA5      44.907   POINT (-72.3243 42.09165)
\end{verbatim}

\begin{enumerate}
\def\labelenumi{\arabic{enumi}.}
\item
  \textbf{GEO\_ID}: This is a unique identifier for each geographic
  unit, corresponding to a standardized code that includes the country
  and ZIP code area.
\item
  \textbf{ZCTA5}: This stands for ``ZIP Code Tabulation Area (ZCTA)''.
  ZCTAs are generalized representations of ZIP codes used by the U.S.
  Census Bureau.
\item
  \textbf{NAME}: This column appears to repeat the ZCTA5 code, so it
  simply be a label for each ZIP code area, duplicating the ZIP code for
  easy reference.
\item
  \textbf{LSAD}: ``Legal/Statistical Area Description''. ``ZCTA5'' in
  this column indicates that the area is a 5-digit ZIP Code Tabulation
  Area.
\item
  \textbf{CENSUSAREA}: This column provides the area of the ZIP Code
  Tabulation Area in square miles or square kilometers, depending on the
  dataset's unit.
\item
  \textbf{geometry}: This column contains geometric data that defines
  the centroid of each ZIP code area. Each entry in this column is a
  ``POINT'' with coordinates (longitude, latitude), representing the
  central location of the ZIP code area.
\end{enumerate}

\subsection{2.}\label{section-4}

\begin{Shaded}
\begin{Highlighting}[]
\ImportTok{import}\NormalTok{ geopandas }\ImportTok{as}\NormalTok{ gpd}

\CommentTok{\# Load the shapefile for all ZIP codes}
\NormalTok{zips\_all\_centroids }\OperatorTok{=}\NormalTok{ gpd.read\_file(}\StringTok{"/Users/xiadizhe/Documents/GitHub/problem{-}set{-}4{-}kevin{-}dylan/gz\_2010\_us\_860\_00\_500k/gz\_2010\_us\_860\_00\_500k.shp"}\NormalTok{)}

\CommentTok{\# Define ZIP code prefixes for Texas (two{-}digit)}
\NormalTok{texas\_prefixes }\OperatorTok{=}\NormalTok{ (}\StringTok{\textquotesingle{}75\textquotesingle{}}\NormalTok{, }\StringTok{\textquotesingle{}76\textquotesingle{}}\NormalTok{, }\StringTok{\textquotesingle{}77\textquotesingle{}}\NormalTok{, }\StringTok{\textquotesingle{}78\textquotesingle{}}\NormalTok{, }\StringTok{\textquotesingle{}79\textquotesingle{}}\NormalTok{)}

\CommentTok{\# Define ZIP code prefixes for bordering states (could be two or three digits)}
\NormalTok{bordering\_states\_prefixes }\OperatorTok{=}\NormalTok{ [}
    \StringTok{\textquotesingle{}73\textquotesingle{}}\NormalTok{, }\StringTok{\textquotesingle{}74\textquotesingle{}}\NormalTok{,       }\CommentTok{\# 73{-}74 }
    \StringTok{\textquotesingle{}870\textquotesingle{}}\NormalTok{, }\StringTok{\textquotesingle{}871\textquotesingle{}}\NormalTok{, }\StringTok{\textquotesingle{}872\textquotesingle{}}\NormalTok{, }\StringTok{\textquotesingle{}873\textquotesingle{}}\NormalTok{, }\StringTok{\textquotesingle{}874\textquotesingle{}}\NormalTok{, }\StringTok{\textquotesingle{}875\textquotesingle{}}\NormalTok{, }\StringTok{\textquotesingle{}876\textquotesingle{}}\NormalTok{, }\StringTok{\textquotesingle{}877\textquotesingle{}}\NormalTok{, }\StringTok{\textquotesingle{}878\textquotesingle{}}\NormalTok{, }\StringTok{\textquotesingle{}879\textquotesingle{}}\NormalTok{,}
\StringTok{\textquotesingle{}880\textquotesingle{}}\NormalTok{, }\StringTok{\textquotesingle{}881\textquotesingle{}}\NormalTok{, }\StringTok{\textquotesingle{}882\textquotesingle{}}\NormalTok{, }\StringTok{\textquotesingle{}883\textquotesingle{}}\NormalTok{, }\StringTok{\textquotesingle{}884\textquotesingle{}}\NormalTok{,  }\CommentTok{\# 870{-}884}
    \StringTok{\textquotesingle{}700\textquotesingle{}}\NormalTok{, }\StringTok{\textquotesingle{}701\textquotesingle{}}\NormalTok{, }\StringTok{\textquotesingle{}702\textquotesingle{}}\NormalTok{, }\StringTok{\textquotesingle{}703\textquotesingle{}}\NormalTok{, }\StringTok{\textquotesingle{}704\textquotesingle{}}\NormalTok{, }\StringTok{\textquotesingle{}705\textquotesingle{}}\NormalTok{, }\StringTok{\textquotesingle{}706\textquotesingle{}}\NormalTok{, }\StringTok{\textquotesingle{}707\textquotesingle{}}\NormalTok{, }\StringTok{\textquotesingle{}708\textquotesingle{}}\NormalTok{, }\StringTok{\textquotesingle{}709\textquotesingle{}}\NormalTok{,}
\StringTok{\textquotesingle{}710\textquotesingle{}}\NormalTok{, }\StringTok{\textquotesingle{}711\textquotesingle{}}\NormalTok{, }\StringTok{\textquotesingle{}712\textquotesingle{}}\NormalTok{, }\StringTok{\textquotesingle{}713\textquotesingle{}}\NormalTok{, }\StringTok{\textquotesingle{}714\textquotesingle{}}\NormalTok{, }\StringTok{\textquotesingle{}715\textquotesingle{}}\NormalTok{, }\StringTok{\textquotesingle{}716\textquotesingle{}}\NormalTok{, }\StringTok{\textquotesingle{}717\textquotesingle{}}\NormalTok{, }\StringTok{\textquotesingle{}718\textquotesingle{}}\NormalTok{, }\StringTok{\textquotesingle{}719\textquotesingle{}}\NormalTok{,}
\StringTok{\textquotesingle{}720\textquotesingle{}}\NormalTok{, }\StringTok{\textquotesingle{}721\textquotesingle{}}\NormalTok{, }\StringTok{\textquotesingle{}722\textquotesingle{}}\NormalTok{, }\StringTok{\textquotesingle{}723\textquotesingle{}}\NormalTok{, }\StringTok{\textquotesingle{}724\textquotesingle{}}\NormalTok{, }\StringTok{\textquotesingle{}725\textquotesingle{}}\NormalTok{, }\StringTok{\textquotesingle{}726\textquotesingle{}}\NormalTok{, }\StringTok{\textquotesingle{}727\textquotesingle{}}\NormalTok{, }\StringTok{\textquotesingle{}728\textquotesingle{}}\NormalTok{, }\StringTok{\textquotesingle{}729\textquotesingle{}}
\StringTok{\textquotesingle{}715\textquotesingle{}}\NormalTok{, }\StringTok{\textquotesingle{}716\textquotesingle{}}\NormalTok{, }\StringTok{\textquotesingle{}717\textquotesingle{}}\NormalTok{, }\StringTok{\textquotesingle{}718\textquotesingle{}}\NormalTok{, }\StringTok{\textquotesingle{}719\textquotesingle{}}\NormalTok{,  }\CommentTok{\# 700{-}715 \& 716{-}729}
\NormalTok{]}

\CommentTok{\# Function to check if a ZIP code matches any of the prefixes}
\KeywordTok{def}\NormalTok{ matches\_prefix(zip\_code, prefixes):}
    \ControlFlowTok{return} \BuiltInTok{any}\NormalTok{(zip\_code.startswith(prefix) }\ControlFlowTok{for}\NormalTok{ prefix }\KeywordTok{in}\NormalTok{ prefixes)}

\CommentTok{\# Filter ZIP codes for Texas}
\NormalTok{zips\_texas\_centroids }\OperatorTok{=}\NormalTok{ zips\_all\_centroids[zips\_all\_centroids[}\StringTok{\textquotesingle{}ZCTA5\textquotesingle{}}\NormalTok{].}\BuiltInTok{apply}\NormalTok{(}\KeywordTok{lambda}\NormalTok{ x: matches\_prefix(x, texas\_prefixes))]}

\CommentTok{\# Filter ZIP codes for Texas and bordering states}
\NormalTok{zips\_texas\_borderstates\_centroids }\OperatorTok{=}\NormalTok{ zips\_all\_centroids[}
\NormalTok{    zips\_all\_centroids[}\StringTok{\textquotesingle{}ZCTA5\textquotesingle{}}\NormalTok{].}\BuiltInTok{apply}\NormalTok{(}\KeywordTok{lambda}\NormalTok{ x: matches\_prefix(x, texas\_prefixes }\OperatorTok{+} \BuiltInTok{tuple}\NormalTok{(bordering\_states\_prefixes)))}
\NormalTok{]}

\CommentTok{\# Get the number of unique ZIP codes in each subset}
\NormalTok{unique\_texas\_zips }\OperatorTok{=}\NormalTok{ zips\_texas\_centroids[}\StringTok{\textquotesingle{}ZCTA5\textquotesingle{}}\NormalTok{].nunique()}
\NormalTok{unique\_borderstate\_zips }\OperatorTok{=}\NormalTok{ zips\_texas\_borderstates\_centroids[}\StringTok{\textquotesingle{}ZCTA5\textquotesingle{}}\NormalTok{].nunique()}

\BuiltInTok{print}\NormalTok{(}\SpecialStringTok{f"Number of unique ZIP codes in Texas: }\SpecialCharTok{\{}\NormalTok{unique\_texas\_zips}\SpecialCharTok{\}}\SpecialStringTok{"}\NormalTok{)}
\BuiltInTok{print}\NormalTok{(}\SpecialStringTok{f"Number of unique ZIP codes in Texas and bordering states: }\SpecialCharTok{\{}\NormalTok{unique\_borderstate\_zips}\SpecialCharTok{\}}\SpecialStringTok{"}\NormalTok{)}
\end{Highlighting}
\end{Shaded}

\begin{verbatim}
Number of unique ZIP codes in Texas: 1935
Number of unique ZIP codes in Texas and bordering states: 4023
\end{verbatim}

\subsection{3.}\label{section-5}

\begin{Shaded}
\begin{Highlighting}[]
\CommentTok{\# Load the 2016 hospital data}
\NormalTok{hospital\_data\_2016 }\OperatorTok{=}\NormalTok{ pd.read\_csv(}\StringTok{\textquotesingle{}/Users/xiadizhe/Documents/GitHub/problem{-}set{-}4{-}kevin{-}dylan/pos2016.csv\textquotesingle{}}\NormalTok{, encoding}\OperatorTok{=}\StringTok{\textquotesingle{}ISO{-}8859{-}1\textquotesingle{}}\NormalTok{)}
\CommentTok{\# Filter hospitals to only include those with PRVDR\_CTGRY\_CD and PRVDR\_CTGRY\_SBTYP\_CD equal to 1}
\NormalTok{hospital\_data\_2016\_filtered }\OperatorTok{=}\NormalTok{ hospital\_data\_2016[}
\NormalTok{    (hospital\_data\_2016[}\StringTok{\textquotesingle{}PRVDR\_CTGRY\_CD\textquotesingle{}}\NormalTok{] }\OperatorTok{==} \DecValTok{1}\NormalTok{) }\OperatorTok{\&} 
\NormalTok{    (hospital\_data\_2016[}\StringTok{\textquotesingle{}PRVDR\_CTGRY\_SBTYP\_CD\textquotesingle{}}\NormalTok{] }\OperatorTok{==} \DecValTok{1}\NormalTok{)}
\NormalTok{]}

\CommentTok{\# Clean up ZIP\_CD column in hospital data}
\NormalTok{hospital\_data\_2016\_filtered[}\StringTok{\textquotesingle{}ZIP\_CD\textquotesingle{}}\NormalTok{] }\OperatorTok{=}\NormalTok{ hospital\_data\_2016\_filtered[}\StringTok{\textquotesingle{}ZIP\_CD\textquotesingle{}}\NormalTok{].}\BuiltInTok{apply}\NormalTok{(}\KeywordTok{lambda}\NormalTok{ x: }\BuiltInTok{str}\NormalTok{(}\BuiltInTok{int}\NormalTok{(}\BuiltInTok{float}\NormalTok{(x))))}

\CommentTok{\# Perform an inner merge to find ZIP codes in Texas and bordering states that contain at least one hospital in 2016}
\NormalTok{zips\_withhospital\_centroids }\OperatorTok{=}\NormalTok{ zips\_texas\_borderstates\_centroids.merge(}
\NormalTok{    hospital\_data\_2016\_filtered[[}\StringTok{\textquotesingle{}ZIP\_CD\textquotesingle{}}\NormalTok{]].drop\_duplicates(),}
\NormalTok{    left\_on}\OperatorTok{=}\StringTok{\textquotesingle{}ZCTA5\textquotesingle{}}\NormalTok{,  }\CommentTok{\# from zip code data}
\NormalTok{    right\_on}\OperatorTok{=}\StringTok{\textquotesingle{}ZIP\_CD\textquotesingle{}}\NormalTok{,  }\CommentTok{\# from hospital data}
\NormalTok{    how}\OperatorTok{=}\StringTok{\textquotesingle{}inner\textquotesingle{}}
\NormalTok{)}

\CommentTok{\# Output the number of unique ZIP codes with at least one hospital in 2016}
\NormalTok{unique\_zip\_count }\OperatorTok{=}\NormalTok{ zips\_withhospital\_centroids[}\StringTok{\textquotesingle{}ZCTA5\textquotesingle{}}\NormalTok{].nunique()}
\BuiltInTok{print}\NormalTok{(}\SpecialStringTok{f"Number of unique ZIP codes with at least one hospital in 2016: }\SpecialCharTok{\{}\NormalTok{unique\_zip\_count}\SpecialCharTok{\}}\SpecialStringTok{"}\NormalTok{)}
\end{Highlighting}
\end{Shaded}

\begin{verbatim}
Number of unique ZIP codes with at least one hospital in 2016: 877
\end{verbatim}

For merge type, it's needed to use an \textbf{inner merge} because we
are only interested in ZIP codes that appear in both
\texttt{zips\_texas\_borderstates\_centroids} and
\texttt{hospital\_data\_2016\_filtered}. This ensures that the ZIP codes
included in the result contain at least one hospital.

The merge is done on the \textbf{ZIP code} variable, specifically using
the \textbf{\texttt{ZCTA5}} column from
\texttt{zips\_texas\_borderstates\_centroids} and the
\textbf{\texttt{ZIP\_CD}} column from
\texttt{hospital\_data\_2016\_filtered}.

\subsection{4.}\label{section-6}

\subsubsection{a.}\label{a.-1}

\begin{Shaded}
\begin{Highlighting}[]
\ImportTok{import}\NormalTok{ time}
\ImportTok{from}\NormalTok{ scipy.spatial }\ImportTok{import}\NormalTok{ KDTree}
\ImportTok{import}\NormalTok{ pandas }\ImportTok{as}\NormalTok{ pd}

\CommentTok{\# Project both datasets to a projected CRS for accurate centroid calculations}
\NormalTok{projected\_texas\_centroids }\OperatorTok{=}\NormalTok{ zips\_texas\_centroids.to\_crs(epsg}\OperatorTok{=}\DecValTok{3857}\NormalTok{)}
\NormalTok{projected\_hospital\_centroids }\OperatorTok{=}\NormalTok{ zips\_withhospital\_centroids.to\_crs(epsg}\OperatorTok{=}\DecValTok{3857}\NormalTok{)}

\CommentTok{\# Select a random sample of 10 ZIP codes from the projected Texas ZIP centroids}
\NormalTok{sample\_zips\_texas }\OperatorTok{=}\NormalTok{ projected\_texas\_centroids.sample(n}\OperatorTok{=}\DecValTok{10}\NormalTok{, random\_state}\OperatorTok{=}\DecValTok{42}\NormalTok{)  }\CommentTok{\# Set a random state for reproducibility}

\CommentTok{\# Start timing}
\NormalTok{start\_time\_sample }\OperatorTok{=}\NormalTok{ time.time()}

\CommentTok{\# Calculate centroids for the sample of Texas ZIP code polygons}
\NormalTok{sample\_texas\_centroids }\OperatorTok{=}\NormalTok{ sample\_zips\_texas[}\StringTok{\textquotesingle{}geometry\textquotesingle{}}\NormalTok{].centroid}
\NormalTok{sample\_texas\_coords }\OperatorTok{=}\NormalTok{ [(point.x, point.y) }\ControlFlowTok{for}\NormalTok{ point }\KeywordTok{in}\NormalTok{ sample\_texas\_centroids]}

\CommentTok{\# Calculate centroids for hospital ZIP code polygons}
\NormalTok{hospital\_centroids }\OperatorTok{=}\NormalTok{ projected\_hospital\_centroids[}\StringTok{\textquotesingle{}geometry\textquotesingle{}}\NormalTok{].centroid}
\NormalTok{hospital\_coords }\OperatorTok{=}\NormalTok{ [(point.x, point.y) }\ControlFlowTok{for}\NormalTok{ point }\KeywordTok{in}\NormalTok{ hospital\_centroids]}

\CommentTok{\# Build a KDTree for hospital coordinates to enable efficient nearest{-}neighbor queries}
\NormalTok{hospital\_tree }\OperatorTok{=}\NormalTok{ KDTree(hospital\_coords)}

\CommentTok{\# Query the nearest hospital for each ZIP code in the sample in one batch}
\NormalTok{sample\_distances, sample\_indices }\OperatorTok{=}\NormalTok{ hospital\_tree.query(sample\_texas\_coords)}

\CommentTok{\# Create a DataFrame with ZIP codes and their corresponding nearest hospital distance}
\NormalTok{results\_sample }\OperatorTok{=}\NormalTok{ pd.DataFrame(\{}
    \StringTok{"ZIP Code"}\NormalTok{: sample\_zips\_texas[}\StringTok{\textquotesingle{}ZCTA5\textquotesingle{}}\NormalTok{],  }\CommentTok{\# Adjust \textquotesingle{}ZCTA5\textquotesingle{} to match the actual ZIP code column name}
    \StringTok{"Distance to Nearest Hospital"}\NormalTok{: sample\_distances}
\NormalTok{\})}

\CommentTok{\# End timing}
\NormalTok{end\_time\_sample }\OperatorTok{=}\NormalTok{ time.time()}
\NormalTok{elapsed\_time\_sample }\OperatorTok{=}\NormalTok{ end\_time\_sample }\OperatorTok{{-}}\NormalTok{ start\_time\_sample}

\CommentTok{\# Estimate the total time for all ZIP codes in zips\_texas\_centroids}
\NormalTok{num\_total\_zips }\OperatorTok{=} \BuiltInTok{len}\NormalTok{(zips\_texas\_centroids)}
\NormalTok{estimated\_total\_time }\OperatorTok{=}\NormalTok{ (elapsed\_time\_sample }\OperatorTok{/} \DecValTok{10}\NormalTok{) }\OperatorTok{*}\NormalTok{ num\_total\_zips}

\CommentTok{\# Print results}
\BuiltInTok{print}\NormalTok{(}\SpecialStringTok{f"Time taken for 10 sample ZIP codes: }\SpecialCharTok{\{}\NormalTok{elapsed\_time\_sample}\SpecialCharTok{:.2f\}}\SpecialStringTok{ seconds"}\NormalTok{)}
\BuiltInTok{print}\NormalTok{(}\SpecialStringTok{f"Estimated total time for all ZIP codes: }\SpecialCharTok{\{}\NormalTok{estimated\_total\_time}\SpecialCharTok{:.2f\}}\SpecialStringTok{ seconds"}\NormalTok{)}
\BuiltInTok{print}\NormalTok{(results\_sample.head())  }\CommentTok{\# Display first few rows of the sample results}
\end{Highlighting}
\end{Shaded}

\begin{verbatim}
Time taken for 10 sample ZIP codes: 0.03 seconds
Estimated total time for all ZIP codes: 5.81 seconds
      ZIP Code  Distance to Nearest Hospital
10413    77580                  13932.112849
24872    75686                      0.000000
24878    75708                      0.000000
28346    75496                  20392.677020
26824    75116                   4609.613031
\end{verbatim}

\subsubsection{b.}\label{b.-1}

\begin{Shaded}
\begin{Highlighting}[]
\ImportTok{import}\NormalTok{ time}
\ImportTok{from}\NormalTok{ scipy.spatial }\ImportTok{import}\NormalTok{ KDTree}
\ImportTok{import}\NormalTok{ pandas }\ImportTok{as}\NormalTok{ pd}

\CommentTok{\# Set control variable}
\NormalTok{run\_full\_calculation }\OperatorTok{=} \VariableTok{True}  \CommentTok{\# Set to True to run the full calculation}

\CommentTok{\# If set to True, perform the full calculation}
\ControlFlowTok{if}\NormalTok{ run\_full\_calculation:}
    \CommentTok{\# Start timing for the full calculation}
\NormalTok{    start\_time\_full }\OperatorTok{=}\NormalTok{ time.time()}

    \CommentTok{\# Re{-}project Texas ZIP code polygons and hospital ZIP code polygons to EPSG:3857 (meters)}
\NormalTok{    projected\_texas\_centroids }\OperatorTok{=}\NormalTok{ zips\_texas\_centroids.to\_crs(epsg}\OperatorTok{=}\DecValTok{3857}\NormalTok{)}
\NormalTok{    projected\_hospital\_centroids }\OperatorTok{=}\NormalTok{ zips\_withhospital\_centroids.to\_crs(epsg}\OperatorTok{=}\DecValTok{3857}\NormalTok{)}

    \CommentTok{\# Calculate centroids for Texas ZIP code polygons}
\NormalTok{    texas\_centroids }\OperatorTok{=}\NormalTok{ projected\_texas\_centroids[}\StringTok{\textquotesingle{}geometry\textquotesingle{}}\NormalTok{].centroid}
\NormalTok{    texas\_coords }\OperatorTok{=}\NormalTok{ [(point.x, point.y) }\ControlFlowTok{for}\NormalTok{ point }\KeywordTok{in}\NormalTok{ texas\_centroids]}

    \CommentTok{\# Calculate centroids for hospital ZIP code polygons}
\NormalTok{    hospital\_centroids }\OperatorTok{=}\NormalTok{ projected\_hospital\_centroids[}\StringTok{\textquotesingle{}geometry\textquotesingle{}}\NormalTok{].centroid}
\NormalTok{    hospital\_coords }\OperatorTok{=}\NormalTok{ [(point.x, point.y) }\ControlFlowTok{for}\NormalTok{ point }\KeywordTok{in}\NormalTok{ hospital\_centroids]}

    \CommentTok{\# Build a KDTree for hospital coordinates to enable efficient nearest{-}neighbor queries}
\NormalTok{    hospital\_tree }\OperatorTok{=}\NormalTok{ KDTree(hospital\_coords)}

    \CommentTok{\# Query the nearest hospital for each Texas ZIP code in one batch}
\NormalTok{    distances, indices }\OperatorTok{=}\NormalTok{ hospital\_tree.query(texas\_coords)}

    \CommentTok{\# Create a DataFrame with ZIP codes and their corresponding nearest hospital distance}
\NormalTok{    results\_full }\OperatorTok{=}\NormalTok{ pd.DataFrame(\{}
        \StringTok{"ZIP Code"}\NormalTok{: projected\_texas\_centroids[}\StringTok{\textquotesingle{}ZCTA5\textquotesingle{}}\NormalTok{],  }\CommentTok{\# Adjust \textquotesingle{}ZCTA5\textquotesingle{} to match the actual ZIP code column name}
        \StringTok{"Distance to Nearest Hospital"}\NormalTok{: distances}
\NormalTok{    \})}

    \CommentTok{\# End timing for the full calculation}
\NormalTok{    end\_time\_full }\OperatorTok{=}\NormalTok{ time.time()}
\NormalTok{    elapsed\_time\_full }\OperatorTok{=}\NormalTok{ end\_time\_full }\OperatorTok{{-}}\NormalTok{ start\_time\_full}

    \CommentTok{\# Print results}
    \BuiltInTok{print}\NormalTok{(}\SpecialStringTok{f"Time taken for the full calculation: }\SpecialCharTok{\{}\NormalTok{elapsed\_time\_full}\SpecialCharTok{:.2f\}}\SpecialStringTok{ seconds"}\NormalTok{)}
    \BuiltInTok{print}\NormalTok{(results\_full.head())  }\CommentTok{\# Display first few rows of the results}

\ControlFlowTok{else}\NormalTok{:}
    \CommentTok{\# Skip the full calculation part}
    \BuiltInTok{print}\NormalTok{(}\StringTok{"Skipping full calculation in preview mode."}\NormalTok{)}
\end{Highlighting}
\end{Shaded}

\begin{verbatim}
Time taken for the full calculation: 0.63 seconds
     ZIP Code  Distance to Nearest Hospital
9207    78624                      0.000000
9208    78626                      0.000000
9209    78628                  14229.181500
9210    78631                  42377.472898
9211    78632                  18272.542866
\end{verbatim}

The time it took to run the entire dataset was about 0.65 seconds, which
is actually much faster than the estimated 5 seconds, most likely
because it takes time to open and run the program itself, and when the
program is running it takes time very quickly, so the estimated time
will be quite different from the actual time

\subsubsection{c.}\label{c.-1}

Units: EPSG:3857 (meters), we need to change the meters into miles.

\begin{Shaded}
\begin{Highlighting}[]
\NormalTok{METERS\_TO\_MILES }\OperatorTok{=} \FloatTok{0.00062}  \CommentTok{\# This is an estimation from meters to miles}

\CommentTok{\# Convert distances from degrees to miles and add as a new column in the DataFrame}
\NormalTok{results\_full[}\StringTok{\textquotesingle{}Distance\_Miles\textquotesingle{}}\NormalTok{] }\OperatorTok{=}\NormalTok{ results\_full[}\StringTok{\textquotesingle{}Distance to Nearest Hospital\textquotesingle{}}\NormalTok{] }\OperatorTok{*}\NormalTok{ METERS\_TO\_MILES}
\end{Highlighting}
\end{Shaded}

\subsection{5.}\label{section-7}

\subsubsection{a.}\label{a.-2}

Right now, the distance is based on the Column of Distance\_Miles, so
the unit is miles.

\subsubsection{b.}\label{b.-2}

\begin{Shaded}
\begin{Highlighting}[]
\CommentTok{\# Calculate the average distance in miles}
\NormalTok{average\_distance\_miles }\OperatorTok{=} \BuiltInTok{sum}\NormalTok{(results\_full[}\StringTok{\textquotesingle{}Distance\_Miles\textquotesingle{}}\NormalTok{]) }\OperatorTok{/} \BuiltInTok{len}\NormalTok{(results\_full[}\StringTok{\textquotesingle{}Distance\_Miles\textquotesingle{}}\NormalTok{])}

\CommentTok{\# Display the average distance}
\BuiltInTok{print}\NormalTok{(}\SpecialStringTok{f"Average distance to the nearest hospital: }\SpecialCharTok{\{}\NormalTok{average\_distance\_miles}\SpecialCharTok{:.2f\}}\SpecialStringTok{ miles"}\NormalTok{)}

\CommentTok{\# Evaluate if the value makes sense}
\ControlFlowTok{if}\NormalTok{ average\_distance\_miles }\OperatorTok{\textless{}} \DecValTok{5}\NormalTok{:}
    \BuiltInTok{print}\NormalTok{(}\StringTok{"This average distance seems quite low, suggesting that most ZIP codes are close to a hospital."}\NormalTok{)}
\ControlFlowTok{elif}\NormalTok{ average\_distance\_miles }\OperatorTok{\textless{}} \DecValTok{15}\NormalTok{:}
    \BuiltInTok{print}\NormalTok{(}\StringTok{"This average distance is reasonable, indicating that hospitals are within moderate proximity for most ZIP codes."}\NormalTok{)}
\ControlFlowTok{else}\NormalTok{:}
    \BuiltInTok{print}\NormalTok{(}\StringTok{"This average distance is quite high, which might suggest limited hospital access in some areas."}\NormalTok{)}
\end{Highlighting}
\end{Shaded}

\begin{verbatim}
Average distance to the nearest hospital: 9.57 miles
This average distance is reasonable, indicating that hospitals are within moderate proximity for most ZIP codes.
\end{verbatim}

\subsubsection{c.}\label{c.-2}

\begin{Shaded}
\begin{Highlighting}[]
\ImportTok{import}\NormalTok{ matplotlib.pyplot }\ImportTok{as}\NormalTok{ plt}
\ImportTok{import}\NormalTok{ matplotlib.colors }\ImportTok{as}\NormalTok{ mcolors}
\ImportTok{import}\NormalTok{ geopandas }\ImportTok{as}\NormalTok{ gpd}

\CommentTok{\# Ensure \textquotesingle{}ZIP Code\textquotesingle{} column is of string type for compatibility with spatial joins}
\NormalTok{results\_full[}\StringTok{\textquotesingle{}ZIP Code\textquotesingle{}}\NormalTok{] }\OperatorTok{=}\NormalTok{ results\_full[}\StringTok{\textquotesingle{}ZIP Code\textquotesingle{}}\NormalTok{].astype(}\BuiltInTok{str}\NormalTok{)}

\CommentTok{\# Drop \textquotesingle{}ZIP Code\textquotesingle{} and \textquotesingle{}Distance\_Miles\textquotesingle{} columns if they already exist in zips\_texas\_centroids to avoid conflicts}
\ControlFlowTok{if} \StringTok{\textquotesingle{}ZIP Code\textquotesingle{}} \KeywordTok{in}\NormalTok{ zips\_texas\_centroids.columns:}
\NormalTok{    zips\_texas\_centroids }\OperatorTok{=}\NormalTok{ zips\_texas\_centroids.drop(columns}\OperatorTok{=}\NormalTok{[}\StringTok{\textquotesingle{}ZIP Code\textquotesingle{}}\NormalTok{])}
\ControlFlowTok{if} \StringTok{\textquotesingle{}Distance\_Miles\textquotesingle{}} \KeywordTok{in}\NormalTok{ zips\_texas\_centroids.columns:}
\NormalTok{    zips\_texas\_centroids }\OperatorTok{=}\NormalTok{ zips\_texas\_centroids.drop(columns}\OperatorTok{=}\NormalTok{[}\StringTok{\textquotesingle{}Distance\_Miles\textquotesingle{}}\NormalTok{])}

\CommentTok{\# Perform the merge}
\NormalTok{zips\_texas\_centroids }\OperatorTok{=}\NormalTok{ zips\_texas\_centroids.merge(}
\NormalTok{    results\_full[[}\StringTok{\textquotesingle{}ZIP Code\textquotesingle{}}\NormalTok{, }\StringTok{\textquotesingle{}Distance\_Miles\textquotesingle{}}\NormalTok{]], }
\NormalTok{    left\_on}\OperatorTok{=}\StringTok{\textquotesingle{}ZCTA5\textquotesingle{}}\NormalTok{, right\_on}\OperatorTok{=}\StringTok{\textquotesingle{}ZIP Code\textquotesingle{}}\NormalTok{, how}\OperatorTok{=}\StringTok{\textquotesingle{}left\textquotesingle{}}
\NormalTok{)}

\CommentTok{\# Print columns to verify \textquotesingle{}Distance\_Miles\textquotesingle{} is added}
\BuiltInTok{print}\NormalTok{(}\StringTok{"Columns in zips\_texas\_centroids after merge:"}\NormalTok{, zips\_texas\_centroids.columns)}

\CommentTok{\# Check if \textquotesingle{}Distance\_Miles\textquotesingle{} exists before plotting}
\ControlFlowTok{if} \StringTok{\textquotesingle{}Distance\_Miles\textquotesingle{}} \KeywordTok{in}\NormalTok{ zips\_texas\_centroids.columns:}
    \CommentTok{\# Define a red color map with adjusted normalization for more contrast in lighter areas}
\NormalTok{    vmin, vmax }\OperatorTok{=} \DecValTok{0}\NormalTok{, }\DecValTok{60}  \CommentTok{\# Adjust the maximum value to make the colors darker overall}
\NormalTok{    norm }\OperatorTok{=}\NormalTok{ mcolors.Normalize(vmin}\OperatorTok{=}\NormalTok{vmin, vmax}\OperatorTok{=}\NormalTok{vmax)}

    \CommentTok{\# Plot the Texas ZIP codes and color by \textquotesingle{}Distance\_Miles\textquotesingle{}}
\NormalTok{    fig, ax }\OperatorTok{=}\NormalTok{ plt.subplots(}\DecValTok{1}\NormalTok{, }\DecValTok{1}\NormalTok{, figsize}\OperatorTok{=}\NormalTok{(}\DecValTok{6}\NormalTok{, }\DecValTok{6}\NormalTok{))}
\NormalTok{    zips\_texas\_centroids.plot(column}\OperatorTok{=}\StringTok{\textquotesingle{}Distance\_Miles\textquotesingle{}}\NormalTok{, }
\NormalTok{                              cmap}\OperatorTok{=}\StringTok{\textquotesingle{}Reds\textquotesingle{}}\NormalTok{,  }\CommentTok{\# Use a red color map}
\NormalTok{                              legend}\OperatorTok{=}\VariableTok{True}\NormalTok{, }
\NormalTok{                              ax}\OperatorTok{=}\NormalTok{ax,}
\NormalTok{                              legend\_kwds}\OperatorTok{=}\NormalTok{\{}\StringTok{\textquotesingle{}label\textquotesingle{}}\NormalTok{: }\StringTok{"Distance to Nearest Hospital (Miles)"}\NormalTok{, }\StringTok{\textquotesingle{}orientation\textquotesingle{}}\NormalTok{: }\StringTok{"horizontal"}\NormalTok{\},}
\NormalTok{                              norm}\OperatorTok{=}\NormalTok{norm)}
\NormalTok{    ax.set\_title(}\StringTok{"Average Distance to Nearest Hospital for ZIP Codes in Texas"}\NormalTok{)}
\NormalTok{    ax.set\_axis\_off()}

\NormalTok{    plt.show()}
\ControlFlowTok{else}\NormalTok{:}
    \BuiltInTok{print}\NormalTok{(}\StringTok{"Error: \textquotesingle{}Distance\_Miles\textquotesingle{} column not found in zips\_texas\_centroids."}\NormalTok{)}
\end{Highlighting}
\end{Shaded}

\begin{verbatim}
Columns in zips_texas_centroids after merge: Index(['GEO_ID', 'ZCTA5', 'NAME', 'LSAD', 'CENSUSAREA', 'geometry', 'ZIP Code',
       'Distance_Miles'],
      dtype='object')
\end{verbatim}

\includegraphics{pset4_template_files/figure-pdf/cell-18-output-2.pdf}

\subsection{Effects of closures on access in Texas (15
pts)}\label{effects-of-closures-on-access-in-texas-15-pts}

\begin{enumerate}
\def\labelenumi{\arabic{enumi}.}
\tightlist
\item
\end{enumerate}

\begin{Shaded}
\begin{Highlighting}[]
\ImportTok{import}\NormalTok{ pandas }\ImportTok{as}\NormalTok{ pd}

\CommentTok{\# list file paths}
\NormalTok{files }\OperatorTok{=}\NormalTok{ \{}
    \DecValTok{2016}\NormalTok{: }\StringTok{\textquotesingle{}/Users/xiadizhe/Documents/GitHub/problem{-}set{-}4{-}kevin{-}dylan/pos2016.csv\textquotesingle{}}\NormalTok{,}
    \DecValTok{2017}\NormalTok{: }\StringTok{\textquotesingle{}/Users/xiadizhe/Documents/GitHub/problem{-}set{-}4{-}kevin{-}dylan/pos2017.csv\textquotesingle{}}\NormalTok{,}
    \DecValTok{2018}\NormalTok{: }\StringTok{\textquotesingle{}/Users/xiadizhe/Documents/GitHub/problem{-}set{-}4{-}kevin{-}dylan/pos2018.csv\textquotesingle{}}\NormalTok{,}
    \DecValTok{2019}\NormalTok{: }\StringTok{\textquotesingle{}/Users/xiadizhe/Documents/GitHub/problem{-}set{-}4{-}kevin{-}dylan/pos2019.csv\textquotesingle{}}
\NormalTok{\}}

\CommentTok{\# Retore data by year}
\NormalTok{data\_by\_year }\OperatorTok{=}\NormalTok{ \{\}}

\CommentTok{\# Upload data by year and filter short{-}term hospitals}
\ControlFlowTok{for}\NormalTok{ year, }\BuiltInTok{file} \KeywordTok{in}\NormalTok{ files.items():}
\NormalTok{    data }\OperatorTok{=}\NormalTok{ pd.read\_csv(}\BuiltInTok{file}\NormalTok{, encoding}\OperatorTok{=}\StringTok{\textquotesingle{}ISO{-}8859{-}1\textquotesingle{}}\NormalTok{)}
\NormalTok{    data\_filtered }\OperatorTok{=}\NormalTok{ data[(data[}\StringTok{\textquotesingle{}PRVDR\_CTGRY\_CD\textquotesingle{}}\NormalTok{] }\OperatorTok{==} \DecValTok{1}\NormalTok{) }\OperatorTok{\&}\NormalTok{ (data[}\StringTok{\textquotesingle{}PRVDR\_CTGRY\_SBTYP\_CD\textquotesingle{}}\NormalTok{] }\OperatorTok{==} \DecValTok{1}\NormalTok{)]}
\NormalTok{    data\_filtered[}\StringTok{\textquotesingle{}Year\textquotesingle{}}\NormalTok{] }\OperatorTok{=}\NormalTok{ year  }
\NormalTok{    data\_by\_year[year] }\OperatorTok{=}\NormalTok{ data\_filtered[[}\StringTok{\textquotesingle{}PRVDR\_NUM\textquotesingle{}}\NormalTok{, }\StringTok{\textquotesingle{}FAC\_NAME\textquotesingle{}}\NormalTok{, }\StringTok{\textquotesingle{}CITY\_NAME\textquotesingle{}}\NormalTok{, }\StringTok{\textquotesingle{}STATE\_CD\textquotesingle{}}\NormalTok{, }\StringTok{\textquotesingle{}PGM\_TRMNTN\_CD\textquotesingle{}}\NormalTok{, }\StringTok{\textquotesingle{}ZIP\_CD\textquotesingle{}}\NormalTok{, }\StringTok{\textquotesingle{}Year\textquotesingle{}}\NormalTok{]]}

\CommentTok{\# combine data}
\NormalTok{df\_combine }\OperatorTok{=}\NormalTok{ pd.concat(data\_by\_year.values())}

\CommentTok{\# Filter all closed hopitals}
\NormalTok{tx\_closures }\OperatorTok{=}\NormalTok{ df\_combine[(df\_combine[}\StringTok{\textquotesingle{}STATE\_CD\textquotesingle{}}\NormalTok{] }\OperatorTok{==} \StringTok{\textquotesingle{}TX\textquotesingle{}}\NormalTok{) }\OperatorTok{\&}\NormalTok{ (df\_combine[}\StringTok{\textquotesingle{}PGM\_TRMNTN\_CD\textquotesingle{}}\NormalTok{] }\OperatorTok{!=} \DecValTok{0}\NormalTok{)]}

\CommentTok{\# Conclude closure times by zid code}
\NormalTok{zip\_code\_closures }\OperatorTok{=}\NormalTok{ tx\_closures[}\StringTok{\textquotesingle{}ZIP\_CD\textquotesingle{}}\NormalTok{].value\_counts().reset\_index()}
\NormalTok{zip\_code\_closures.columns }\OperatorTok{=}\NormalTok{ [}\StringTok{\textquotesingle{}ZIP\_CD\textquotesingle{}}\NormalTok{, }\StringTok{\textquotesingle{}closure\_count\textquotesingle{}}\NormalTok{]}

\CommentTok{\# Indicate the outcomes}
\NormalTok{closure\_summary }\OperatorTok{=}\NormalTok{ zip\_code\_closures[}\StringTok{\textquotesingle{}closure\_count\textquotesingle{}}\NormalTok{].value\_counts().reset\_index()}
\NormalTok{closure\_summary.columns }\OperatorTok{=}\NormalTok{ [}\StringTok{\textquotesingle{}closure\_count\textquotesingle{}}\NormalTok{, }\StringTok{\textquotesingle{}num\_zip\_codes\textquotesingle{}}\NormalTok{]}

\CommentTok{\# Print the outcomes}
\BuiltInTok{print}\NormalTok{(}\StringTok{"Directly affected Texas ZIP codes with closure count:"}\NormalTok{)}
\BuiltInTok{print}\NormalTok{(zip\_code\_closures)}
\BuiltInTok{print}\NormalTok{(}\StringTok{"}\CharTok{\textbackslash{}n}\StringTok{Number of zip codes vs. number of closures they experienced:"}\NormalTok{)}
\BuiltInTok{print}\NormalTok{(closure\_summary)}
\end{Highlighting}
\end{Shaded}

\begin{verbatim}
Directly affected Texas ZIP codes with closure count:
      ZIP_CD  closure_count
0    79902.0             26
1    77054.0             19
2    75235.0             17
3    77338.0             13
4    76648.0             12
..       ...            ...
338  78336.0              1
339  75390.0              1
340  78613.0              1
341  75087.0              1
342  77090.0              1

[343 rows x 2 columns]

Number of zip codes vs. number of closures they experienced:
    closure_count  num_zip_codes
0               4            243
1               8             50
2              12             17
3               1              7
4               3              6
5               5              5
6               2              4
7               7              3
8              11              2
9              26              1
10             19              1
11             17              1
12             13              1
13             10              1
14              6              1
\end{verbatim}

\begin{enumerate}
\def\labelenumi{\arabic{enumi}.}
\setcounter{enumi}{1}
\tightlist
\item
\end{enumerate}

\begin{Shaded}
\begin{Highlighting}[]
\ImportTok{import}\NormalTok{ geopandas }\ImportTok{as}\NormalTok{ gpd}
\ImportTok{import}\NormalTok{ pandas }\ImportTok{as}\NormalTok{ pd}
\ImportTok{import}\NormalTok{ matplotlib.pyplot }\ImportTok{as}\NormalTok{ plt}

\CommentTok{\# File path for SHP file}
\NormalTok{shapefile\_path }\OperatorTok{=} \StringTok{"/Users/xiadizhe/Documents/GitHub/problem{-}set{-}4{-}kevin{-}dylan/gz\_2010\_us\_860\_00\_500k/gz\_2010\_us\_860\_00\_500k.shp"}
\NormalTok{gdf }\OperatorTok{=}\NormalTok{ gpd.read\_file(shapefile\_path)}

\CommentTok{\# Convert ZCTA5 to string}
\NormalTok{gdf[}\StringTok{\textquotesingle{}ZCTA5\textquotesingle{}}\NormalTok{] }\OperatorTok{=}\NormalTok{ gdf[}\StringTok{\textquotesingle{}ZCTA5\textquotesingle{}}\NormalTok{].astype(}\BuiltInTok{str}\NormalTok{)}

\CommentTok{\# Filter Texas zip codes}
\NormalTok{tx\_gdf }\OperatorTok{=}\NormalTok{ gdf[gdf[}\StringTok{\textquotesingle{}ZCTA5\textquotesingle{}}\NormalTok{].}\BuiltInTok{str}\NormalTok{.startswith((}\StringTok{\textquotesingle{}75\textquotesingle{}}\NormalTok{, }\StringTok{\textquotesingle{}76\textquotesingle{}}\NormalTok{, }\StringTok{\textquotesingle{}77\textquotesingle{}}\NormalTok{, }\StringTok{\textquotesingle{}78\textquotesingle{}}\NormalTok{, }\StringTok{\textquotesingle{}79\textquotesingle{}}\NormalTok{))]}

\CommentTok{\# Apply zip\_code\_closures to filter zip codes with at least one closure}
\NormalTok{affected\_zip\_codes }\OperatorTok{=}\NormalTok{ zip\_code\_closures[zip\_code\_closures[}\StringTok{\textquotesingle{}closure\_count\textquotesingle{}}\NormalTok{] }\OperatorTok{\textgreater{}} \DecValTok{0}\NormalTok{]}
\NormalTok{affected\_zip\_codes[}\StringTok{\textquotesingle{}ZIP\_CD\textquotesingle{}}\NormalTok{] }\OperatorTok{=}\NormalTok{ affected\_zip\_codes[}\StringTok{\textquotesingle{}ZIP\_CD\textquotesingle{}}\NormalTok{].}\BuiltInTok{apply}\NormalTok{(}\KeywordTok{lambda}\NormalTok{ x: }\BuiltInTok{str}\NormalTok{(}\BuiltInTok{int}\NormalTok{(}\BuiltInTok{float}\NormalTok{(x))))}

\CommentTok{\# Merge data}
\NormalTok{merged\_gdf }\OperatorTok{=}\NormalTok{ tx\_gdf.merge(affected\_zip\_codes, left\_on}\OperatorTok{=}\StringTok{\textquotesingle{}ZCTA5\textquotesingle{}}\NormalTok{, right\_on}\OperatorTok{=}\StringTok{\textquotesingle{}ZIP\_CD\textquotesingle{}}\NormalTok{, how}\OperatorTok{=}\StringTok{\textquotesingle{}inner\textquotesingle{}}\NormalTok{)}

\CommentTok{\# Filter unnecessary geographic data}
\NormalTok{merged\_gdf }\OperatorTok{=}\NormalTok{ merged\_gdf[merged\_gdf.geometry.notnull() }\OperatorTok{\&}\NormalTok{ merged\_gdf.is\_valid]}

\CommentTok{\# Draw the map with color shades based on closure count}
\NormalTok{fig, ax }\OperatorTok{=}\NormalTok{ plt.subplots(}\DecValTok{1}\NormalTok{, }\DecValTok{1}\NormalTok{, figsize}\OperatorTok{=}\NormalTok{(}\DecValTok{10}\NormalTok{, }\DecValTok{10}\NormalTok{))}
\NormalTok{merged\_gdf.plot(column}\OperatorTok{=}\StringTok{\textquotesingle{}closure\_count\textquotesingle{}}\NormalTok{, cmap}\OperatorTok{=}\StringTok{\textquotesingle{}OrRd\textquotesingle{}}\NormalTok{, linewidth}\OperatorTok{=}\FloatTok{0.8}\NormalTok{, ax}\OperatorTok{=}\NormalTok{ax, edgecolor}\OperatorTok{=}\StringTok{\textquotesingle{}0.8\textquotesingle{}}\NormalTok{, legend}\OperatorTok{=}\VariableTok{True}\NormalTok{)}
\NormalTok{ax.set\_title(}\StringTok{"Texas Zip Codes Affected by Hospital Closures (2016{-}2019)"}\NormalTok{)}
\NormalTok{ax.set\_axis\_off()}

\CommentTok{\# Show the plot}
\NormalTok{plt.show()}

\CommentTok{\# Print the number of directly affected zip codes}
\NormalTok{affected\_zip\_count }\OperatorTok{=}\NormalTok{ merged\_gdf[}\StringTok{\textquotesingle{}ZCTA5\textquotesingle{}}\NormalTok{].nunique()}
\BuiltInTok{print}\NormalTok{(}\SpecialStringTok{f"There are }\SpecialCharTok{\{}\NormalTok{affected\_zip\_count}\SpecialCharTok{\}}\SpecialStringTok{ directly affected zip codes in Texas."}\NormalTok{)}
\end{Highlighting}
\end{Shaded}

\includegraphics{pset4_template_files/figure-pdf/cell-20-output-1.pdf}

\begin{verbatim}
There are 334 directly affected zip codes in Texas.
\end{verbatim}

\begin{enumerate}
\def\labelenumi{\arabic{enumi}.}
\setcounter{enumi}{2}
\tightlist
\item
\end{enumerate}

\begin{Shaded}
\begin{Highlighting}[]
\ImportTok{import}\NormalTok{ geopandas }\ImportTok{as}\NormalTok{ gpd}
\ImportTok{import}\NormalTok{ pandas }\ImportTok{as}\NormalTok{ pd}

\CommentTok{\# Convert ZIP codes and ensure consistent format}
\NormalTok{directly\_affected\_zip\_codes }\OperatorTok{=}\NormalTok{ zip\_code\_closures[zip\_code\_closures[}\StringTok{\textquotesingle{}closure\_count\textquotesingle{}}\NormalTok{] }\OperatorTok{\textgreater{}} \DecValTok{0}\NormalTok{]}
\NormalTok{directly\_affected\_zip\_codes[}\StringTok{\textquotesingle{}ZIP\_CD\textquotesingle{}}\NormalTok{] }\OperatorTok{=}\NormalTok{ directly\_affected\_zip\_codes[}\StringTok{\textquotesingle{}ZIP\_CD\textquotesingle{}}\NormalTok{].}\BuiltInTok{apply}\NormalTok{(}\KeywordTok{lambda}\NormalTok{ x: }\BuiltInTok{str}\NormalTok{(}\BuiltInTok{int}\NormalTok{(}\BuiltInTok{float}\NormalTok{(x))))}

\CommentTok{\# Create the GeoDataFrame of directly affected ZIP codes}
\NormalTok{directly\_affected\_gdf }\OperatorTok{=}\NormalTok{ tx\_gdf[tx\_gdf[}\StringTok{\textquotesingle{}ZCTA5\textquotesingle{}}\NormalTok{].isin(directly\_affected\_zip\_codes[}\StringTok{\textquotesingle{}ZIP\_CD\textquotesingle{}}\NormalTok{])].copy()}

\CommentTok{\# Create a 10{-}mile buffer around directly affected ZIP codes}
\NormalTok{directly\_affected\_gdf[}\StringTok{\textquotesingle{}geometry\textquotesingle{}}\NormalTok{] }\OperatorTok{=}\NormalTok{ directly\_affected\_gdf.geometry.}\BuiltInTok{buffer}\NormalTok{(}\DecValTok{10} \OperatorTok{*} \FloatTok{1609.34}\NormalTok{)  }

\CommentTok{\# Perform a spatial join to find ZIP codes within the buffer zone}
\NormalTok{indirectly\_affected\_gdf }\OperatorTok{=}\NormalTok{ gpd.sjoin(tx\_gdf, directly\_affected\_gdf, how}\OperatorTok{=}\StringTok{"inner"}\NormalTok{, predicate}\OperatorTok{=}\StringTok{"intersects"}\NormalTok{)}

\CommentTok{\# Remove directly affected ZIP codes to keep only those indirectly affected}
\NormalTok{indirectly\_affected\_zip\_codes }\OperatorTok{=}\NormalTok{ indirectly\_affected\_gdf[}\OperatorTok{\textasciitilde{}}\NormalTok{indirectly\_affected\_gdf[}\StringTok{\textquotesingle{}ZCTA5\_left\textquotesingle{}}\NormalTok{].isin(directly\_affected\_gdf[}\StringTok{\textquotesingle{}ZCTA5\textquotesingle{}}\NormalTok{])]}

\CommentTok{\# Calculate the number of indirectly affected ZIP codes}
\NormalTok{indirectly\_affected\_zip\_count }\OperatorTok{=}\NormalTok{ indirectly\_affected\_zip\_codes[}\StringTok{\textquotesingle{}ZCTA5\_left\textquotesingle{}}\NormalTok{].nunique()}
\BuiltInTok{print}\NormalTok{(}\SpecialStringTok{f"There are }\SpecialCharTok{\{}\NormalTok{indirectly\_affected\_zip\_count}\SpecialCharTok{\}}\SpecialStringTok{ indirectly affected ZIP codes in Texas."}\NormalTok{)}
\end{Highlighting}
\end{Shaded}

\begin{verbatim}
/var/folders/_4/32019gfn3jscfp2y27sc0v180000gn/T/ipykernel_8908/1323952495.py:12: UserWarning:

Geometry is in a geographic CRS. Results from 'buffer' are likely incorrect. Use 'GeoSeries.to_crs()' to re-project geometries to a projected CRS before this operation.

\end{verbatim}

\begin{verbatim}
There are 1601 indirectly affected ZIP codes in Texas.
\end{verbatim}

\begin{enumerate}
\def\labelenumi{\arabic{enumi}.}
\setcounter{enumi}{3}
\tightlist
\item
\end{enumerate}

\begin{Shaded}
\begin{Highlighting}[]
\ImportTok{import}\NormalTok{ geopandas }\ImportTok{as}\NormalTok{ gpd}
\ImportTok{import}\NormalTok{ pandas }\ImportTok{as}\NormalTok{ pd}
\ImportTok{import}\NormalTok{ matplotlib.pyplot }\ImportTok{as}\NormalTok{ plt}

\CommentTok{\# File paths for the shapefile and hospital closure data}
\NormalTok{shapefile\_path }\OperatorTok{=} \StringTok{"/Users/xiadizhe/Documents/GitHub/problem{-}set{-}4{-}kevin{-}dylan/gz\_2010\_us\_860\_00\_500k/gz\_2010\_us\_860\_00\_500k.shp"}
\NormalTok{files }\OperatorTok{=}\NormalTok{ \{}
    \DecValTok{2016}\NormalTok{: }\StringTok{\textquotesingle{}/Users/xiadizhe/Documents/GitHub/problem{-}set{-}4{-}kevin{-}dylan/pos2016.csv\textquotesingle{}}\NormalTok{,}
    \DecValTok{2017}\NormalTok{: }\StringTok{\textquotesingle{}/Users/xiadizhe/Documents/GitHub/problem{-}set{-}4{-}kevin{-}dylan/pos2017.csv\textquotesingle{}}\NormalTok{,}
    \DecValTok{2018}\NormalTok{: }\StringTok{\textquotesingle{}/Users/xiadizhe/Documents/GitHub/problem{-}set{-}4{-}kevin{-}dylan/pos2018.csv\textquotesingle{}}\NormalTok{,}
    \DecValTok{2019}\NormalTok{: }\StringTok{\textquotesingle{}/Users/xiadizhe/Documents/GitHub/problem{-}set{-}4{-}kevin{-}dylan/pos2019.csv\textquotesingle{}}
\NormalTok{\}}

\CommentTok{\# Load shapefile and filter for Texas ZIP codes}
\NormalTok{gdf }\OperatorTok{=}\NormalTok{ gpd.read\_file(shapefile\_path)}
\NormalTok{gdf[}\StringTok{\textquotesingle{}ZCTA5\textquotesingle{}}\NormalTok{] }\OperatorTok{=}\NormalTok{ gdf[}\StringTok{\textquotesingle{}ZCTA5\textquotesingle{}}\NormalTok{].astype(}\BuiltInTok{str}\NormalTok{)}
\NormalTok{tx\_gdf }\OperatorTok{=}\NormalTok{ gdf[gdf[}\StringTok{\textquotesingle{}ZCTA5\textquotesingle{}}\NormalTok{].}\BuiltInTok{str}\NormalTok{.startswith((}\StringTok{\textquotesingle{}75\textquotesingle{}}\NormalTok{, }\StringTok{\textquotesingle{}76\textquotesingle{}}\NormalTok{, }\StringTok{\textquotesingle{}77\textquotesingle{}}\NormalTok{, }\StringTok{\textquotesingle{}78\textquotesingle{}}\NormalTok{, }\StringTok{\textquotesingle{}79\textquotesingle{}}\NormalTok{))]}

\CommentTok{\# Set a common CRS for accurate distance calculations, e.g., UTM zone for Texas}
\NormalTok{tx\_gdf }\OperatorTok{=}\NormalTok{ tx\_gdf.to\_crs(epsg}\OperatorTok{=}\DecValTok{32614}\NormalTok{)}

\CommentTok{\# Load and combine closure data for each year, filtering for short{-}term hospitals}
\NormalTok{data\_by\_year }\OperatorTok{=}\NormalTok{ \{\}}
\ControlFlowTok{for}\NormalTok{ year, }\BuiltInTok{file} \KeywordTok{in}\NormalTok{ files.items():}
\NormalTok{    data }\OperatorTok{=}\NormalTok{ pd.read\_csv(}\BuiltInTok{file}\NormalTok{, encoding}\OperatorTok{=}\StringTok{\textquotesingle{}ISO{-}8859{-}1\textquotesingle{}}\NormalTok{, low\_memory}\OperatorTok{=}\VariableTok{False}\NormalTok{)}
\NormalTok{    data\_filtered }\OperatorTok{=}\NormalTok{ data[(data[}\StringTok{\textquotesingle{}PRVDR\_CTGRY\_CD\textquotesingle{}}\NormalTok{] }\OperatorTok{==} \DecValTok{1}\NormalTok{) }\OperatorTok{\&}\NormalTok{ (data[}\StringTok{\textquotesingle{}PRVDR\_CTGRY\_SBTYP\_CD\textquotesingle{}}\NormalTok{] }\OperatorTok{==} \DecValTok{1}\NormalTok{)]}
\NormalTok{    data\_filtered.loc[:, }\StringTok{\textquotesingle{}Year\textquotesingle{}}\NormalTok{] }\OperatorTok{=}\NormalTok{ year}
\NormalTok{    data\_by\_year[year] }\OperatorTok{=}\NormalTok{ data\_filtered[[}\StringTok{\textquotesingle{}PRVDR\_NUM\textquotesingle{}}\NormalTok{, }\StringTok{\textquotesingle{}FAC\_NAME\textquotesingle{}}\NormalTok{, }\StringTok{\textquotesingle{}CITY\_NAME\textquotesingle{}}\NormalTok{, }\StringTok{\textquotesingle{}STATE\_CD\textquotesingle{}}\NormalTok{, }\StringTok{\textquotesingle{}PGM\_TRMNTN\_CD\textquotesingle{}}\NormalTok{, }\StringTok{\textquotesingle{}ZIP\_CD\textquotesingle{}}\NormalTok{, }\StringTok{\textquotesingle{}Year\textquotesingle{}}\NormalTok{]]}

\CommentTok{\# Combine data and filter for Texas closures}
\NormalTok{df\_combine }\OperatorTok{=}\NormalTok{ pd.concat(data\_by\_year.values())}
\NormalTok{tx\_closures }\OperatorTok{=}\NormalTok{ df\_combine[(df\_combine[}\StringTok{\textquotesingle{}STATE\_CD\textquotesingle{}}\NormalTok{] }\OperatorTok{==} \StringTok{\textquotesingle{}TX\textquotesingle{}}\NormalTok{) }\OperatorTok{\&}\NormalTok{ (df\_combine[}\StringTok{\textquotesingle{}PGM\_TRMNTN\_CD\textquotesingle{}}\NormalTok{] }\OperatorTok{!=} \DecValTok{0}\NormalTok{)]}

\CommentTok{\# Count closures by ZIP code}
\NormalTok{zip\_code\_closures }\OperatorTok{=}\NormalTok{ tx\_closures[}\StringTok{\textquotesingle{}ZIP\_CD\textquotesingle{}}\NormalTok{].value\_counts().reset\_index()}
\NormalTok{zip\_code\_closures.columns }\OperatorTok{=}\NormalTok{ [}\StringTok{\textquotesingle{}ZIP\_CD\textquotesingle{}}\NormalTok{, }\StringTok{\textquotesingle{}closure\_count\textquotesingle{}}\NormalTok{]}

\CommentTok{\# Identify directly affected ZIP codes}
\NormalTok{directly\_affected\_zip\_codes }\OperatorTok{=}\NormalTok{ zip\_code\_closures[zip\_code\_closures[}\StringTok{\textquotesingle{}closure\_count\textquotesingle{}}\NormalTok{] }\OperatorTok{\textgreater{}} \DecValTok{0}\NormalTok{]}
\NormalTok{directly\_affected\_zip\_codes[}\StringTok{\textquotesingle{}ZIP\_CD\textquotesingle{}}\NormalTok{] }\OperatorTok{=}\NormalTok{ directly\_affected\_zip\_codes[}\StringTok{\textquotesingle{}ZIP\_CD\textquotesingle{}}\NormalTok{].}\BuiltInTok{apply}\NormalTok{(}\KeywordTok{lambda}\NormalTok{ x: }\BuiltInTok{str}\NormalTok{(}\BuiltInTok{int}\NormalTok{(}\BuiltInTok{float}\NormalTok{(x))))}

\CommentTok{\# Create GeoDataFrame of directly affected ZIP codes}
\NormalTok{directly\_affected\_gdf }\OperatorTok{=}\NormalTok{ tx\_gdf[tx\_gdf[}\StringTok{\textquotesingle{}ZCTA5\textquotesingle{}}\NormalTok{].isin(directly\_affected\_zip\_codes[}\StringTok{\textquotesingle{}ZIP\_CD\textquotesingle{}}\NormalTok{])].copy()}
\NormalTok{directly\_affected\_gdf[}\StringTok{\textquotesingle{}category\textquotesingle{}}\NormalTok{] }\OperatorTok{=} \StringTok{\textquotesingle{}Directly Affected\textquotesingle{}}

\CommentTok{\# Create a 10{-}mile buffer around directly affected ZIP codes}
\NormalTok{directly\_affected\_gdf[}\StringTok{\textquotesingle{}geometry\textquotesingle{}}\NormalTok{] }\OperatorTok{=}\NormalTok{ directly\_affected\_gdf.geometry.}\BuiltInTok{buffer}\NormalTok{(}\DecValTok{10} \OperatorTok{*} \FloatTok{1609.34}\NormalTok{)  }\CommentTok{\# 10 miles in meters}

\CommentTok{\# Perform a spatial join to find indirectly affected ZIP codes within the buffer}
\NormalTok{indirectly\_affected\_gdf }\OperatorTok{=}\NormalTok{ gpd.sjoin(tx\_gdf, directly\_affected\_gdf, how}\OperatorTok{=}\StringTok{"inner"}\NormalTok{, predicate}\OperatorTok{=}\StringTok{"intersects"}\NormalTok{)}
\NormalTok{indirectly\_affected\_gdf }\OperatorTok{=}\NormalTok{ indirectly\_affected\_gdf[}\OperatorTok{\textasciitilde{}}\NormalTok{indirectly\_affected\_gdf[}\StringTok{\textquotesingle{}ZCTA5\_left\textquotesingle{}}\NormalTok{].isin(directly\_affected\_gdf[}\StringTok{\textquotesingle{}ZCTA5\textquotesingle{}}\NormalTok{])]}
\NormalTok{indirectly\_affected\_gdf[}\StringTok{\textquotesingle{}category\textquotesingle{}}\NormalTok{] }\OperatorTok{=} \StringTok{\textquotesingle{}Indirectly Affected\textquotesingle{}}

\CommentTok{\# Create a GeoDataFrame for unaffected ZIP codes}
\NormalTok{not\_affected\_gdf }\OperatorTok{=}\NormalTok{ tx\_gdf[}\OperatorTok{\textasciitilde{}}\NormalTok{tx\_gdf[}\StringTok{\textquotesingle{}ZCTA5\textquotesingle{}}\NormalTok{].isin(directly\_affected\_zip\_codes[}\StringTok{\textquotesingle{}ZIP\_CD\textquotesingle{}}\NormalTok{]) }\OperatorTok{\&} 
                          \OperatorTok{\textasciitilde{}}\NormalTok{tx\_gdf[}\StringTok{\textquotesingle{}ZCTA5\textquotesingle{}}\NormalTok{].isin(indirectly\_affected\_gdf[}\StringTok{\textquotesingle{}ZCTA5\_left\textquotesingle{}}\NormalTok{])]}
\NormalTok{not\_affected\_gdf[}\StringTok{\textquotesingle{}category\textquotesingle{}}\NormalTok{] }\OperatorTok{=} \StringTok{\textquotesingle{}Not Affected\textquotesingle{}}

\CommentTok{\# Combine all three categories into a single GeoDataFrame}
\NormalTok{final\_gdf }\OperatorTok{=}\NormalTok{ pd.concat([}
\NormalTok{    directly\_affected\_gdf[[}\StringTok{\textquotesingle{}ZCTA5\textquotesingle{}}\NormalTok{, }\StringTok{\textquotesingle{}geometry\textquotesingle{}}\NormalTok{, }\StringTok{\textquotesingle{}category\textquotesingle{}}\NormalTok{]],}
\NormalTok{    indirectly\_affected\_gdf[[}\StringTok{\textquotesingle{}ZCTA5\_left\textquotesingle{}}\NormalTok{, }\StringTok{\textquotesingle{}geometry\textquotesingle{}}\NormalTok{, }\StringTok{\textquotesingle{}category\textquotesingle{}}\NormalTok{]].rename(columns}\OperatorTok{=}\NormalTok{\{}\StringTok{\textquotesingle{}ZCTA5\_left\textquotesingle{}}\NormalTok{: }\StringTok{\textquotesingle{}ZCTA5\textquotesingle{}}\NormalTok{\}),}
\NormalTok{    not\_affected\_gdf[[}\StringTok{\textquotesingle{}ZCTA5\textquotesingle{}}\NormalTok{, }\StringTok{\textquotesingle{}geometry\textquotesingle{}}\NormalTok{, }\StringTok{\textquotesingle{}category\textquotesingle{}}\NormalTok{]]}
\NormalTok{])}

\CommentTok{\# Ensure the final\_gdf is also in the same CRS for consistency}
\NormalTok{final\_gdf }\OperatorTok{=}\NormalTok{ final\_gdf.to\_crs(epsg}\OperatorTok{=}\DecValTok{32614}\NormalTok{)}

\CommentTok{\# Plot the choropleth map with different colors for each category}
\NormalTok{fig, ax }\OperatorTok{=}\NormalTok{ plt.subplots(}\DecValTok{1}\NormalTok{, }\DecValTok{1}\NormalTok{, figsize}\OperatorTok{=}\NormalTok{(}\DecValTok{12}\NormalTok{, }\DecValTok{12}\NormalTok{))}
\NormalTok{final\_gdf.plot(column}\OperatorTok{=}\StringTok{\textquotesingle{}category\textquotesingle{}}\NormalTok{, cmap}\OperatorTok{=}\StringTok{\textquotesingle{}Set1\textquotesingle{}}\NormalTok{, linewidth}\OperatorTok{=}\FloatTok{0.8}\NormalTok{, ax}\OperatorTok{=}\NormalTok{ax, edgecolor}\OperatorTok{=}\StringTok{\textquotesingle{}0.8\textquotesingle{}}\NormalTok{, legend}\OperatorTok{=}\VariableTok{True}\NormalTok{, categorical}\OperatorTok{=}\VariableTok{True}\NormalTok{)}

\CommentTok{\# Set title and remove axis}
\NormalTok{ax.set\_title(}\StringTok{"Texas ZIP Codes Affected by Hospital Closures (2016{-}2019)"}\NormalTok{)}
\NormalTok{ax.set\_axis\_off()}

\CommentTok{\# Show plot}
\NormalTok{plt.show()}
\end{Highlighting}
\end{Shaded}

\includegraphics{pset4_template_files/figure-pdf/cell-22-output-1.pdf}

\subsection{Reflecting on the exercise (10
pts)}\label{reflecting-on-the-exercise-10-pts}

\subsubsection{1. Partner 1:}\label{partner-1}

There would be misunderstanding of temporary closures and permanent
closures. Some hospitals close temporarily for various reasons and then
reopen. A simple method may mistake these for permanent closures. In
addition, there would be misunderstanding of changes in ownership or
merger. Hospitals that merge or change ownership might be listed as
different names in the dataset. The ``first-pass'' approach could
mistakenly classify these cases as closures.

In order to avoid these potential problems, we can do multi-year
analysis. We can make sure that whether a hospital is closing or not by
look at how it has been operated over several years. If it has been
closed for a few years, it is probably permanent closing. Meanwhile, we
can also analyze from cross-sectional databases. We can check other
public records for more accurate data on hospital openings and closures.
Such as, procurement data from major suppliers and tax payment record.

\subsubsection{2. Partner 2:}\label{partner-2}

We identify zip codes affected by hospital closures by observing the
termination status of hospitals within each zip code between different
years. This approach provides a straightforward way to detect potential
gaps in healthcare accessibility caused by closures. However, it only
captures the presence or absence of hospitals without accounting for the
proximity of hospitals outside the zip code boundaries.

Besides, it is reasonable to worry about our assumption that any zip
code with a hospital closure will experiences a reduction in hospital
resource immediately. But in real world, especially in urban areas,
multiple hospital oftern serve overlapping regions.

Improvements: 1. Take Population Density and Demographics into weight
calculation process; 2. Use Travel Time Instead of Distance, which will
focus on convinience rather than pure distance; 3. Analyze Hospital
Capacity Changes




\end{document}
